\documentclass[12pt]{article}
\usepackage{amsmath}
\begin{document}


Cosmic rays are atomic nuclei with velocities close to the speed of light. Ultra-high energy cosmic rays (UHECRs) have approximately ten million times more energetic than the most extreme particles produced at the Large Hadron Collider and likely originate from nearby extragalactic sources. Important astrophysical questions include: what phenomenon accelerates particles to such large energies, which astronomical objects host the accelerators, and what sorts of nuclei end up being energized? We develop a multilevel Bayesian framework for assessing the association of UHECRs and a candidate source population that allows us to model
1) the source luminosities,
2) the marked Poisson point process for the UHECR directions and energies,
3) the magnetic deflection, and
4) the detection measurement error and the observatory's nonuniform exposure.

Our multilevel statistical model can use astrophysical components anywhere on a spectrum from simplicity to great complexity and realism.  Currently, we are using astrophysical submodels typical of those in the literature, but our results suggest using more complex alternative models in future work. The principal obstacle to computing with this framework is the combinatorial explosion in the number of possible associations as the sizes of the candidate source population and the UHECR sample grow. We present a Markov chain Monte Carlo algorithm for estimating the parameters and computing, via Chib's method, the marginal likelihoods used in model comparison. We implement our methodology to the 69 UHECRs observed by Pierre Auger Observatory during the years 2004--2009,
and a volume-complete catalog of 17 nearby active galactic nuclei.


\end{document}
