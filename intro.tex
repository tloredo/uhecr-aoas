\section{Introduction}

Cosmic ray particles are naturally produced, positively charged atomic nuclei
arriving from outer space with velocities close to the speed of light.  Their
celestial origin was discovered at the beginning of the 20th century using
electrometers hoisted by balloons into the upper atmosphere \cite{Hess12}. 
Nearly a century's subsequent observations have determined that the impinging
flux includes particles with energies spanning a dozen or more orders of
magnitude \cite{Cronin99}.

The origin of cosmic rays is not well-understood. The Lorentz force
experienced by a charged particle in a magnetic field alters its trajectory.
Simple estimates imply that cosmic rays with energy $E \lta 10^{15}$~eV 
have trajectories so strongly bent
by the Galactic magnetic field that they are largely trapped within the
Galaxy.\footnote{We follow the standard astronomical convention of using
``Galaxy'' and ``Galactic'' to refer to the Milky Way galaxy.}  The
acceleration sites and the source populations are not definitively known but
probably include supernovae, pulsars, stars with strong winds, and
stellar-mass black holes. For recent reviews, see \cite{Cronin99,Hillas06}.
More mysterious, however, are the highest energy
cosmic rays.

In the 1960s, large arrays of cosmic ray detectors began detecting cosmic rays
with enormous energies; by 1991 a few events were seen with energies $\sim
100$~EeV (where 1~EeV $=10^{18}$~eV). In the 1990's cosmic ray astronomers
built huge detector arrays specifically targeting these energetic events.  The
largest and most recent of these were the Akeno Giant Air Shower Array (AGASA;
\cite{1992APh.....1...27C}) and the High Resolution Fly's Eye (HiRes;
\cite{2002NIMPA.482..457B}); each detected a few dozen cosmic rays with
$E>10$~EeV, so-called ultra-high energy cosmic rays (UHECRs). For recent
reviews, see \cite{KO11-UHECRs,LS11-UHECRs}.  The sources of
these cosmic rays likely form a relatively nearby extragalactic population. On
the one hand their trajectories are only weakly deflected by galactic magnetic
fields so they are unconfined to the galaxy from which they originate.  On the
other hand, they are unlikely to reach us from distant (and thus isotropically
distributed) cosmological sources.  Cosmic rays with energies above the
Greisen-Zatsepin-Kuzmin (GZK) limit of $\sim 50$~EeV  should scatter off of
cosmic microwave background photons, losing some of their energy to pion
production with each interaction
\cite{G66-GZK,ZK66-GZK}. Thus the universe is not
transparent to UHECRs; they are not expected to travel more than about 50 to
100 megaparsecs (Mpc) before their energies fall below the GZK limit. 
Detectable UHECRs likely emanate from relatively nearby extragalactic sources.
 Notably, over this cosmologically modest distance scale there is significant
anisotropy in the distribution of matter that should be reflected in the
arrival directions of UHECRs (albeit distorted by magnetic deflection). 
Astronomers hope that continued study of the directions and energies of UHECRs
will address the fundamental questions of the field: What phenomenon
accelerates particles to such large energies?  Which astronomical objects host
the accelerators?  What sorts of nuclei end up being energized?  In
addition, UHECRs serve as probes of galactic and intergalactic magnetic
fields.

The flux of UHECRs is very small, approximately 1 per square kilometer per
century for energies $E\gta 50$ EeV.  Large detectors are needed to find
these elusive objects.  The construction of the largest and most sensitive
detector to date, the Pierre Auger Observatory (PAO) \cite{PAO04-Proto}
in Argentina, has marked a significant step towards that goal.  The
observatory uses a combination of air fluorescence telescopes and water
Cerenkov surface detectors to observe the air shower generated when a cosmic
ray interacts with nuclei in the upper atmosphere over the
observatory.  The surface detectors (SDs) operate continuously, detecting
energetic subatomic particles produced in the air shower and reaching the
ground.  The fluorescence detectors (FDs) image light from the air shower and
supplement the surface detector data for events detected on clear, dark
nights.\footnote{The FD on-time is about 13\% \cite{PAO10-GZK}, but analysis
can reveal complications preventing use of the data---e.g., obscuration due to
light cloud cover, or showers with significant development underground---so
fewer than 13\% of events have usable FD data.  These few so-called {\em
hybrid} events are important for calibrating energy measurements and provide
information about cosmic ray composition vs.\ energy.}
PAO began taking data in 2004 even as it was under construction; it reached
its baseline design in June 2008 with an array of $\approx 1600$ SDs
covering approximately $3000$ km$^2$, surrounded by four
fluorescence telescope stations (with six telescopes in each station)
observing the atmosphere over the array.

By 31 August 2007, PAO had detected 81 UHECRs with $E > 40$~EeV
(see \cite{PAO07-Aniso}, hereafter PAO-07).  A major finding is clear evidence of
an energy cutoff resembling the predicted GZK cutoff, i.e., a sharp drop in
the energy spectrum above $\approx 100$~EeV and a discernable pile-up of
events at energies below that \cite{PAO10-GZK}. This supports the idea that
the UHECRs must originate in the nearby universe.

The PAO team searched for correlations between the cosmic
ray arrival directions and the directions to nearby active galactic nuclei
(AGN) (initial results were reported in PAO-07; further details
and a catalog of the events are in \cite{PAO08-AGN}, hereafter PAO-08).
AGN are unusually bright cores of galaxies; there is strong (but
necessarily indirect) evidence that they are the sites of supermassive black
holes rapidly accreting nearby gas and stars and ejecting some of the
accreting material in an energetic, jet-like outflow.  AGN are theoretically
favored sites for producing UHECRs; electromagnetic observations indicate
particles are accelerated to high energies near AGN.  The PAO team's analysis
was based on a significance test that counted the number of UHECRs with
best-fit directions within a critical angle, $\psi$, of an active galactic
nucleus (AGN) in a catalog of local AGN candidate sources (more details
about the catalog appear below); the number was compared with what would be
expected from an isotropic UHECR directional distribution using a \pval.  A
simple sequential approach was adopted.  The earliest half of the data was
used to tune three parameters defining the test statistic by minimizing the
\pval; the parameters were the critical angle $\psi$; a distance cutoff,
$\Dmax$, such that only AGN closer than $\Dmax$ were considered possible
hosts; and an energy threshold, $\Eth$, with only UHECRs with $E>\Eth$
considered to be associated with AGN.  With these parameters tuned
($\Eth=56$~EeV, $\psi=3.1^\circ$, $\Dmax=75$~Mpc), the test was applied to
the later half of the data; 13 UHECRs in that period had $E>\Eth$.  The
resulting \pval\ of $1.7\times 10^{-3}$ was taken as indicating the data
justify rejecting the hypothesis of isotropic arrival directions ``with at
least a 99\% confidence level.''  The PAO team was careful to note that this
result did not necessarily imply that UHECRs were associated with the
cataloged AGN, but rather that they were likely to be associated with some
nearby extragalactic population with similar anisotropy.

Along with these results, the PAO team published a catalog of energy and
direction estimates for the 27 UHECRs satisfying the $E>\Eth$ criterion,
including both the earliest 14 events used to define $\Eth$, and the 13
subsequent events used to obtain the reported \pval\ (the PAO data are
proprietary; measurements of the other 54 events used in the analysis were
not published).  Their exciting but admittedly suggestive statistical result
spurred a number of subsequent analyses of these early published PAO UHECR
arrival directions, adopting different methods and aiming to make more
specific claims about the hosts of the UHECRs.  Roughly speaking, these
analyses found similarly suggestive evidence for anisotropy, but no
conclusive evidence for any specific association hypothesis.

% If needed, cite studies cited by WMJ11.

In late 2010, the PAO team published a revised catalog, including new data
collected through 2009 (\cite{PAO10-AnisoUpdate}; hereafter PAO-10).  An
improved analysis pipeline revised the energies of earlier events downward
by 1~EeV; accordingly, the team adopted $\Eth = 55$~EeV on the new energy
scale.  The new catalog includes measurements of 42 new UHECRs (with $E>\Eth$)
detected from 1~September 2009 through 31~December 2010, for a total of 69
events.  A repeat of the previous analysis (adding the new events but again
excluding the early tuning events) produced a larger \pval\ of
$3\times 10^{-3}$, i.e., {\em weaker} evidence against the isotropic
hypothesis.  The team performed a number of other analyses (including
considering new candidate host populations).  Despite the growth of the
post-tuning sample size from 14 to 55, they found evidence for anisotropy
weakened, across diverse analyses.  Time-resolved measures of anisotropy
provided puzzling indications that later data might have different directional
properties than early data, although the sample size is too small to
conclusively demonstrate this.

Here we describe a new framework for modeling UHECR data based on Bayesian
multilevel modeling of cosmic ray emission, propagation, and detection.
Astrophysically, a virtue of this approach is that physical and experimental
processes have explicit representations in the framework, facilitating
exploration of various scientific hypotheses potentially explaining the
data, and physical interpretation of the results.  This is in contrast to
hypothesis testing approaches where elements such as angular and energy
thresholds only implicitly represent underlying physics, and potentially
conflate astrophysical and experimental effects (e.g., magnetic scattering
of trajectories, and measurement errors in direction).
Statistically, one important virtue of our framework is the ability to
handle a priori uncertainty in model parameters via marginalization. 
Marginalization also accounts for the uncertainty in such parameters via
weighted averaging, rather than fixing them at precise, tuned values.  This
eliminates the need to tune energy, angle, and distance scales with a subset
of the data that must then be excluded from a final analysis.  Such parameters
are allowed to adapt to the data, but the ``Ockham's razor'' effect associated
with marginalization penalizes models for fine-tuned degrees of freedom,
thereby accounting for the adaptation.

In this paper we describe our general framework, computational algorithms
for its implementation, and results from analyses based on a few
representative models.  Our models are somewhat simplistic astrophysically,
although similar to models adopted in previous studies.  We do not aim to
reach final conclusions about the sources of UHECRs; the focus here is on
developing new methodology and demonstrating the capabilities of the
approach in the context of simple models.

An important finding is that {\em thorough and accurate independent analysis
of the PAO data likely requires more data than has so far been publicly
released} by the PAO collaboration.  In particular, although our Bayesian
approach eliminates the need for tuning, in the absence of publicly available
``untuned'' data (i.e., measurements of lower-energy cosmic rays), we cannot
completely eliminate the effects of tuning from analyses of the
published data (Bayesian or otherwise).  Additionally, a Bayesian analysis can
(and should) use event-by-event (i.e., heteroskedastic) measurement
uncertainties, but these are not publicly available.  Finally, astrophysically
plausible conclusions about the sources of UHECRs will require models more
sophisticated than those we explore here (and those explored in other recent
studies). The PAO observatory continues to operate; we hope future PAO
publications will include more complete event catalogs, enabling more
thoroughgoing analyses than are presently possible.
