\section{Cen~A single-source model}
\label{app:CenA}

Recent PAO results on the chemical composition of UHECRs, cited above,
suggest that UHECRs may be predominantly heavy nuclei, which would
suffer large magnetic deflections.  Some investigators have suggested
that UHECRs are all heavy nuclei from a single source---the nearest AGN,
Cen~A---with the apparent approximate isotropy of arrival directions
a consequence of strong deflection \cite{B+09-CenA,GBdS10-CenA,BdS12-CenA}.

As a simple test of this idea, we studied a model corresponding to our
buckshot AGN association model, but with a single candidate source, Cen~A, and
with the association fraction $f=1$.  Figure~\ref{fig:BF-CenA} shows the Bayes
factor comparing such a model to the isotropic background model, conditional
on $\kappa$, for various partitions of the data.  For $\kappa=0$, the Cen~A
model makes the same predictions as the background model, and the Bayes factor
is unity.  As $\kappa$ increases, the Bayes factor never grows significantly
larger than unity, so the Cen~A model is not supported by any partition of the
data, for any $\kappa$.  For partitions of the data including the 42 Period~3
events, the Bayes factor opposes the Cen~A model, strongly ruling out models
with $\kappa\simgreat 0.5$, i.e., with deflection angular scales $<90^\circ$.

\begin{figure}
\centerline{\includegraphics[angle=-90,width=\textwidth]{BF-CenAvsIsotropic-SmallKappa.eps}}
\caption{Bayes factors comparing a model Cen~A to be the source of
all UHECRs against the null isotropic background model, conditional
on $\kappa$, shown as a function of $\kappa$.  Results are
shown for various partions of the data (identified by line style,
identified in the legend).  Right panel expands the low-$\kappa$ region.}
\label{fig:BF-CenA}
\end{figure}

The $90^\circ$ scale is consistent with the expected angular scale for
deflection by a regular magnetic field in equation~(\ref{dflxn-reg}) as long
as $Z$ is large ($\simgreat 15$); but note that this equation is valid only in
the small-deflection limit.  The $90^\circ$ scale is consistent with turbulent
magnetic deflection only for the heaviest nuclei (e.g., for iron $Z=26$) and
for large magnetic field and length scales.  For these astrophysically
plausible parameters, the Cen~A model remains disfavored, although not
overwhelmingly.  We thus consider the PAO-10 data to argue against the
hypothesis that all UHECRs originate from Cen~A.
