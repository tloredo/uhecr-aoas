\section{Description of cosmic ray and candidate host data}

%..............................................................................
\subsection{Cosmic ray data}
\label{sec:data}

The reported PAO measurements depend not only on the intrinsic particle
population but also on many experimental and algorithmic choices in the
detection and analysis chain, many of them associated with the need to
distinguish between events of interest and background events from uninteresting
but uncontrollable sources (e.g., natural radioactivity).  UHECRs can
impinge on the observatory at any time, from any direction and with any
energy.
However, virtually no background sources
produce events with properties mimicking those of very high energy cosmic rays
arriving from directions well above the horizon.
Cosmic rays with $E>3$~EeV arriving from any direction lying within a large
window on the sky create air showers detected with nearly 100\% efficiency
(no false positives, no false dismissals).  The SDs and FDs measure
the spatio-temporal development of the air shower which allows
the energy and arrival direction to be measured.  The uncertainties depend
upon how many counters of each type are triggered plus the systematic and
statistical uncertainties implicit in modeling  the development of the
air shower. The PAO team reports energy and
arrival direction estimates for each cosmic ray falling within the geometric
bounds of its zone of secure detection.\footnote{The directional criterion
adopted for the PAO catalogs is that an event is reported if it best-fit
arrival direction is within $60^\circ$ of the observatory's zenith, the
local normal to Earth's surface at the time of the event.}

We consider the $N_C = 69$ UHECRs with energies $E \geq \Eth = 55$~EeV
cataloged in PAO-10, which reports measurements of all UHECRs seen by PAO
through 31 December 2009 with $E \geq \Eth$, based on analysis of the
surface detector data only.  Although our framework does not tune an event
selection criterion, for interpreting the results it is important to
remember that the $\Eth=55$~EeV threshold value was set to maximize a
signature of anisotropy in an early subset of the data.
The tuning data included the 14 earliest
reported events, detected from 1~January 2004 to 26~May 2006 (inclusive;
Period~1), as well as numerous unreported events with $E<\Eth$.  The first
published catalog in PAO-08 included 13 subsequent UHECRs observed through 31
August 2007 (Period~2).  The PAO-10 catalog includes 42 additional UHECRs
observed through 31 December 2009 (Period~3). Table~1 in PAO-10 provides
information about the three periods, including the sky exposure for each
period, which is not simply proportional to duration (the observatory grew in
size considerably through 2008).  Data for cosmic rays with $E < \Eth$ are not
publicly available.\footnote{The PAO web site hosts public data for 1\% of
lower-energy cosmic rays, but the sample is not statistically characterized
and UHECRs are not included.}

The direction estimate for a particular cosmic ray is the result of a
complicated analysis of time series data from the array of PAO surface
detectors.\footnote{The SD data may be supplemented by data from the
fluorescence detectors for hybrid events observed under favorable
conditions, but there are very few such events at ultra-high energies,
and PAO-10 reports analysis of SD data only.}
Roughly speaking, the direction is inferred by triangulation. 
The analysis produces a likelihood function for the cosmic ray arrival
direction, $\omega$ (a unit vector on the celestial sphere).  The shapes of
the likelihood contours are not simple, but they are roughly azimuthally
symmetric about the best-fit direction.  The PAO-10 catalog summarizes the
likelihood function with a best-fit direction, and a typical directional
uncertainty of $\approx 0.9^\circ$ corresponding to the angular radius of an
azimuthally symmetric 68.3\% confidence region.  We use these summaries to
approximate the likelihood functions with a Fisher distribution with mode at
the best-fit direction for each cosmic ray, and with concentration parameter
$\kappa_c = 9323$, corresponding to a 68.3\% confidence region with an
angular radius of 0.9$^\circ$.  Let $d_i$ denote the data associated with
cosmic ray $i$, and $\omega_i$ denote its actual arrival direction  (an
unknown parameter).  The
likelihood function for the direction is
\ba
\ell(\omega_i)
  \coloneqq P(d_i|\omega_i)
   \approx \frac{\kappa_c}{4\pi\sinh(\kappa_c)} \exp(\kappa_c n_i\cdot\omega_i),
\label{ell-def}
\ea
where $n_i$ denotes the best-fit direction for cosmic ray $i$ (a function
of the observed data), and we have
scaled the likelihood function so its integral over $\omega_i$ is unity,
merely as a convenient convention.  Bonifazi et al.\
\cite{B+PAO09-DrxnUncert} provide more information about the PAO direction
measurement capability.  Note that the expected angular scale of magnetic
deflection is larger than the PAO directional uncertainties, significantly
so if UHECRs are heavy nuclei (see \S~\ref{sec:dflxn}).

Similarly, the analysis pipeline produces energy estimates for each event.
These estimates have significant random and systematic uncertainties.
The PAO-10 catalog reports a best-fit energy for each reported event,
with a typical energy resolution of $\approx 15$\% \cite{PAO10-GZK}.
This uncertainty is the random uncertainty in estimates based on surface
detector data.\footnote{For hybrid events, the uncertainty can be better than
6\%.  The percentages indicate the sizes of the root-mean-square errors of
energy estimates in calibration experiments; the errors appear approximately
normally distributed and have some dependence on cosmic ray composition
\cite{PAO08-GZK}.}
The energy scale of UHECRs is so far beyond the scales of easily measured
natural phenomena that an accurate absolute energy calibration for UHECRs
requires significant extrapolation of particle physics theory to regimes not
explored directly by experiment, and this is a significant source of
systematic (correlated) uncertainty in the measured energies.  A key feature
of PAO is the presence of both surface and fluorescence detectors at the same
site; this enables a more secure calibration than was possible for earlier
experiments that relied on a single type of detector.  Analysis and modeling
of FD data produces energy estimates with an absolute energy scale systematic
uncertainty of 22\%.  The absolute energy scale for the SD analysis is calibrated by
comparison of the SD and FD estimates for hybrid events, and the resulting
systematic uncertainty in the SD energy scale with respect to the FD scale is
$\approx 10$\%.

The models we study here do not make use of the reported energies and are
unaffected by these uncertainties.  But our framework readily generalizes to
account for energy dependence.  In principle it is straightforward to account
for the random uncertainties, but a consistent treatment requires data for
events below any imposed threshold:  the true energies of events with best-fit
energies below threshold could be above threshold (and vice versa for those
with best-fit energies above threshold); accounting for this requires data to
energies below astrophysically important thresholds.  The systematic
uncertainties become important for joint analyses of PAO data with data from
other experiments, and for linking results of spectral analyses to particle
physics theory.


%..............................................................................
\subsection{Candidate source catalog}

As candidate sources for the PAO UHECRs, the analysis reported in PAO-07 and
PAO-08, and several subsequent analyses, considered 694 
AGN within $\approx 75$~Mpc from the 12th catalog 
assembled by V\'eron-Cetty and V\'eron \cite{VCV-12thAGNCat} (VCV). 
This catalog includes data on all AGN and quasars (AGN with star-like
images) with published spectroscopic redshifts; it includes observations
from numerous investigators using diverse equipment and AGN selection
methods, and does not represent a statistically well-characterized sample of
AGN.\footnote{VCV say of the catalog, ``This catalogue should not be used
for any statistical analysis as it is not complete in any sense, except that
it is, we hope, a complete survey of the literature.''}
Subsequent analyses in PAO-10, and a few other analyses, used more recent
catalogs of active galaxies or normal galaxies, including flux-limited
catalogs (i.e., well-characterized catalogs that contain all bright sources
within a specified volume, but dimmer sources only in progressively smaller
volumes).

%As noted above, the Greisen-Zatsepin-Kuzmin (GZK) limit implies that cosmic
%rays with energies $\gtrsim$ 50~Eev should interact with cosmic microwave
%background photons and should almost never reach the earth from distance
%longer than 50~Mpc.

For the representative analyses reported here, we consider the 17 AGN
cataloged by Goulding et al.\cite{2010MNRAS.406..597G} (2010; hereafter G10)
as candidate sources.  This is a well-characterized {\em volume}-limited
sample; it includes all infrared-bright AGN within 15~Mpc.  
For each AGN in the calalog, we take its position on the sky, $\varpi_k$
($k=1$ to $\Nsrc$), and its distance, $D_k$, to be known precisely (galaxy
directions have negligible uncertainties compared to cosmic ray directions).
Notably, this catalog includes Centaurus~A (Cen~A), the nearest AGN
($D\approx 4.0$~Mpc), an unusually active and morphologically peculiar AGN.
Theorists have hypothesized Cen~A to be a source of many or even most UHECRs
if UHECRs are heavy nuclei, which would be deflected through large angles;
see \cite{B+09-CenA,GBdS10-CenA,BdS12-CenA}.  The small size of this catalog
facilitates thorough exploration of our methodology: Markov chain Monte
Carlo algorithms can be validated against more straightforward algorithms
that could not be deployed on large catalogs, and simulation studies are
feasible that would be too computationally expensive with large catalogs.
Also, for simple ``standard candle'' models (adopted here and in other
studies), that assign all sources the same cosmic ray intensity, little is
gained by considering large catalogs, because assigning detectable cosmic
ray intensities to distant sources would imply cosmic ray fluxes from nearby
sources too large to be compatible with the data.

We also include an isotropic background component as a ``zeroth" source.
This allows a model to assign some UHECRs to sources not included in the AGN
catalog (either galaxies not cataloged, or other, unobserved sources). In
addition, we consider an isotropic source distribution for {\em all} cosmic
rays (i.e., a model with only the zeroth source) as a ``null'' model for
comparison with models that associate some cosmic rays with AGN or other
discrete sources.  An isotropic distribution is convenient for calculations
and has been adopted as a null hypothesis in several previous studies. 
Historically, before PAO's convincing observation of a GZK-like cutoff in
the UHECR energy spectrum, the isotropic distribution was meant to represent
a distant cosmological origin for UHECRs.  Accepting the null would indicate
that the GZK prediction was incorrect, and that changes in fundamental
physics would be required to explain UHECRs.  In light of PAO's compelling
observation of a GZK-like cutoff (with its implied $\sim 100$~Mpc distance
scale), interpreting an isotropic null or background component is
problematical if there are many light nuclei among the UHECRs.  We adopt it
here both for convenience and due to precedent.  We discuss this further
below.

%..............................................................................
\subsection{Sky maps}

Figures~\ref{fig:skymap} and \ref{fig:skymaps} show sky maps displaying the
directions to both the UHECRs seen by PAO, and the AGN in the G10 catalog.
The directions are shown in an equal-area Hammer-Aitoff projection in
Galactic coordinates; the Galactic plane is the equator (Galactic latitude
$b=0^\circ$), and the vertically-oriented grid lines are meridians of
constant Galactic longitude, $l$.  The star indicates the south celestial
pole (SCP), the direction directly above Earth's south pole (effects like
precession and nutation of the Earth's axis are negligible for this
application and we ignore them in this description).  The thick gray line
bounds the PAO field of view.  The UHECR and AGN directions are displayed as
``tissots,'' projections of circular patches centered on the reported
directions.  The small tissots show the UHECR directions; the tissot size is
$2^\circ$, corresponding to $\approx 2$ standard deviation errors, and the
tissot color indicates energy.  The large green tissots indicate AGN
directions; the tissot size is $5^\circ$, corresponding to a plausible scale
for magnetic deflection of UHE protons in the Galactic magnetic
field.\footnote{If UHECRs are comprised of heavier, more positively charged
nuclei, they could suffer much larger deflections; see \S~\ref{sec:dflxn}.}
The tissots are rendered with transparency; the two darker tissots near the
Galactic north pole indicate pairs of AGN with nearly coincident directions.
Two of the AGN tissots are outlined in solid black; these correspond to the
two nearest AGN, Centaurus~A (Cen~A, also known as NGC~5128) and NGC~4945,
neighboring AGN at distances of 4.0 and 3.9~Mpc (as reported in G10).  Five
others are outlined in dashed black; these have distances ranging from 6.6
to 10.0~Mpc. The remaining 10 AGN have distances from 11.5 to 15.0~Mpc. 
Four of the AGN are outside the PAO field of view, but depending on the
scale of magnetic deflection, they could be sources of observable cosmic
rays.


% Skymaps made with Auger2009+LocalAGN.py script.
% Made PNG versions to preserve transparency; convert to PS Level 3
% binaries with imgtops -8 -3.

\begin{figure}
\begin{centering}
\includegraphics[width=\textwidth]{CR+LocalAGN-all.eps}
\end{centering}
\caption{Sky map showing directions to 69 UHECRs detected by PAO, and
to 17 nearby AGN from the catalog of Goulding et al..  Directions are
shown in an equal-area Hammer-Aitoff projection in Galactic coordinates.
Thick gray line indicates the boundary of the PAO field of view.
Small tissots show UHECR directions; tissot radius is $2^\circ$ corresponding
to $\approx 2$ standard deviation errors; tissot color indicates energy.
Large green tissots indicate AGN directions; tissot radius is $5^\circ$.
Thin curves are geodesics connecting each UHECR to its nearest AGN.
Maps here and in Fig.~\ref{fig:skymaps} show UHECR directions from different
PAO observing periods, as labeled.}
\label{fig:skymap}
\end{figure}

\begin{figure}
\begin{centering}
\includegraphics[width=.9\textwidth]{CR+LocalAGN-1-nobar.eps}\\
\includegraphics[width=.9\textwidth]{CR+LocalAGN-2-nobar.eps}\\
\includegraphics[width=.9\textwidth]{CR+LocalAGN-3.eps}
\end{centering}
\caption{Sky maps showing directions to UHECRs detected by PAO in
each of three observing periods (as labeled).  Symbols are as in
Fig.~\ref{fig:skymap}.}
\label{fig:skymaps}
\end{figure}

Figure~\ref{fig:skymap} shows all 69 UHECR directions; the panels in
figure~\ref{fig:skymaps} show subsets of the UHECRs corresponding to the
three PAO analysis periods described above.  The thin curves (teal) show
geodesics connecting each UHECR to its nearest AGN.  In the full-catalog map,
there is a noticable concentration of cosmic ray directions near the
directions of Cen~A and NGC~4945; a few other AGN also have conspicuously
close cosmic rays.  A concentration in the vicinity of these two closest AGN
is also evident in the maps for periods~1 and 2 (note that the slightly
extended red tissot near the direction of NGC~4945 corresponds to two nearly
coincident UHECRs).  Curiously, except for a single UHECR about $6^\circ$
from NGC~4945, no such concentration is evident in the map for Period~3,
despite it having about three times the number of UHECRs found in earlier
periods.  This is a presage of results from our quantitative analysis that
suggest the data may not be consistent with simple models for the cosmic ray
directions, with or without AGN associations.

%..............................................................................
\subsection{PAO exposure}
\label{sec:expo}

PAO is not equally sensitive to cosmic rays coming from all directions.
Quantitative assessment of evidence for associations or other anisotropy must
account for the observatory's direction-dependent exposure.

Let $F$ be the cosmic ray flux at Earth from a source at a given direction,
$\tdrxn$, i.e., the expected number of cosmic rays per unit time per unit
area normal to $\tdrxn$.   Then the expected number of rays detected in a
short time interval $dt$ is $F \Aperp(t,\tdrxn) dt$, where $\Aperp(t,\tdrxn)$ is
the projected area of the observatory toward $\tdrxn$ at time $t$. The total
expected number of cosmic rays is given by integrating over $t$; it can be
written as $F\expo(\tdrxn)$, with the {\em exposure map} $\expo(\tdrxn)$
defined by
\begin{equation}
\expo(\tdrxn) \coloneqq \int_T  \Aperp(\tdrxn,t) dt;
\label{expo-def-main}
\end{equation}
the integral is over the time intervals when the observatory was operating,
denoted collectively by $T$.
Supplementary Appendix~\ref{app:expo} describes calculation of
$\expo(\tdrxn)$; the thick gray curves shown in the sky maps mark the
boundary of the region of nonzero exposure.
