% Template for the submission to:
%   The Annals of Probability           [aop]
%   The Annals of Applied Probability   [aap]
%   The Annals of Statistics            [aos]
%   The Annals of Applied Statistics    [aoas]
%   Stochastic Systems                  [ssy]
%
%Author: In this template, the places where you need to add information
%        (or delete line) are indicated by {???}.  Mostly the information
%        required is obvious, but some explanations are given in lines starting
%Author:
%All other lines should be ignored.  After editing, there should be
%no instances of ??? after this line.

% use option [preprint] to remove info line at bottom
% journal options: aop,aap,aos,aoas,ssy
% natbib option: authoryear
\documentclass[dvips,aoas]{imsart}

\usepackage{amssymb,amsbsy}
\usepackage{graphicx}

%\usepackage{amsthm,amsmath,natbib}
%\RequirePackage[colorlinks,citecolor=blue,urlcolor=blue]{hyperref}

% provide arXiv number if available:
%\arxiv{math.PR/0000000}

% put your definitions there:
\startlocaldefs
\newcommand{\txw}{\textwidth}

\renewcommand{\arraystretch}{1.2}


\newcommand{\bm}[1]{\mbox{\boldmath{$#1$}}}

\newcommand{\be}{\begin{equation}}
\newcommand{\ee}{\end{equation}}
\newcommand{\ba}{\begin{eqnarray}}
\newcommand{\ea}{\end{eqnarray}}
\newcommand{\barr}{\begin{array}}
\newcommand{\earr}{\end{array}}
\newcommand{\bc}{\begin{center}}
\newcommand{\ec}{\end{center}}

\newcommand{\D}{\overline{\mbox{D}}}

\newcommand{\simless}[0]{\mathbin{\lower 3pt\hbox
   {$\rlap{\raise 5pt\hbox{$\char'074$}}\mathchar"7218$}}}
\newcommand{\simgreat}[0]{\mathbin{\lower 3pt\hbox
   {$\rlap{\raise 5pt\hbox{$\char'076$}}\mathchar"7218$}}}
\newcommand{\gta}[0]{\simgreat}
\newcommand{\lta}[0]{\simless}
\newcommand\msun{{\rm M}_\odot}
\endlocaldefs

\begin{document}

\begin{frontmatter}

% "Title of the paper"
\title{Multilevel Models of the Arrival of the Ultra High Energy Cosmic Rays}
\runtitle{???}

% indicate corresponding author with \corref{}
% \author{\fnms{John} \snm{Smith}\corref{}\ead[label=e1]{smith@foo.com}\thanksref{t1}}
% \thankstext{t1}{Thanks to somebody}
% \address{line 1\\ line 2\\ printead{e1}}
% \affiliation{Some University}

\author{\fnms{???} \snm{???}\ead[label=e1]{???}}
\address{\printead{e1}}
%\and
%\author{\fnms{???} \snm{???}\ead[label=e2]{???}}
%\address{\printead{e2}}
%\affiliation{???}

\runauthor{???}

%\begin{abstract}
%\end{abstract}

%\begin{keyword}[class=AMS]
%\kwd[Primary ]{}
%\kwd{}
%\kwd[; secondary ]{}
%\end{keyword}

%\begin{keyword}
%\kwd{}
%\kwd{}
%\end{keyword}

\end{frontmatter}

\section{Introduction}
\subsection{Cosmic Ray Experiments}

Beginning in the 1990's large area detectors began to
operate (Akeno Giant Air Shower Array [AGASA]; High
Resolution Fly's Eye [HiRes]) and detected small
numbers of ultrahigh energy cosmic rays (UHECRs). These
cosmic rays form a distinctive population because they
are unconfined to the galaxy and likely emanate from
nearby extragalactic sources (within a few $100$ Mpc).
The existence of such energetic entities leads to
fundamental astrophysical questions: what phenomenon
accelerates particles to such large energies, which
astronomical objects host the accelerators and what
sorts of nuclei end up being energized?

The flux of UHECRs is very small, approximately 1 per
square kilometer per century for energies $E>60$ EeV
(where 1 EeV=$10^{18}$ eV). Large detectors are needed
to find these elusive objects.  The construction of
the largest ($\sim 7000$ km$^2$ sr) and most sensitive
detector to date, the Pierre Auger Observatory (PAO) in
Argentina, has marked a significant step towards that
goal.  The observatory uses a combination of air
fluorescence and water Cerenkov counters to detect
air showers generated by the impinging cosmic rays. The
observatory began taking data in 2004 even as it was
under construction; the PAO reached its baseline design in
June 2008.

PAO had observed 69 cosmic rays with $E \gta 55$ EeV
through the end of 2009.  The data depends not only on
the intrinsic astrophysical properties of the UHECRs
but also on many experimental choices in the chain that
leads from detection to measurement. For UHECRs
arriving from a well-defined direction with respect to
the observatory an air shower is detected with nearly
100\% efficiency (no false positives, no false
dismissals).  The air and/or water counters are
sensitive to the development of the air shower which,
in turn, allows the energy and arrival direction to be
measured. The uncertainties depend upon how many
counters of each type are triggered plus the systematic
and statistical uncertainties implicit in the modeling
of the development of the air shower. The observatory
reports the energy and arrival direction for each
cosmic ray falling within the geometric bounds of its
zone of 100\% detection.  It does not report
uncertainties on a cosmic ray-by-cosmic ray basis. It
summarizes these as follows: the absolute energy scale
is uncertain by 22\%, the energy resolution is about
15\% and the angular resolution is better than
$0.9^\circ$.

\subsection{Previous Work}
PAO has collected the data of 69 UHECRs with energy $\geq$ 55 EeV between
January 1, 2004 and Dec 31, 2009. PAO reported the data of these UHECRs in
3 time periods, January 1, 2004 - May 26, 2006, May 27, 2006 - August 31, 2007
and September 1, 2007 - December 31, 2009. In [PAO 2010], the analysis by PAO
is based on the counts of UHECRs with energy $\geq$ 55 EeV and arrival direction
within 3.1$^\circ$ from an AGN in the VCV catalog. The energy and angular distance
cutoff, 55 EeV and 3.1$^\circ$ are chosen by using the data from period 1 to
minimize the probability that the correlation with the AGNs could occur by chance if
the flux is isotropic. The result is that there are 20 correaltion out of 55 events from
periods 2 and 3. The p-value against the null hypothesis of isotropic flux is $3\times 10^{-3}$.

\subsection{Our Work}


\section{Data}
\subsection{AGNs}
According to the Greisen-Zatsepin-Kuz(GZK)
limit, the cosmic rays with energies $\gtrsim$ 50 Eev should interact with cosmic
microwave background photons and should almost never reach the earth from distance
longer than 50 Mpc. In this study, we consider the 17 AGNs in the catalog of
Goulding(2010) as candidate sources. For each AGN $k$ in the calalog, we assume
that its position on the sky $\varpi_k$ and its distance $r_k$ are known precisely.
We also include the isotropic background source as a ``zeroth" source. This allows the model
to assign the UHECRs to sorces not included in the AGN catalog.

\subsection{UHECRs}
We consider the 69 UHECRs with energies $\geq$ 55 EeV. According to PAO, the arrival direction of these UHECRs are measured
with angular uncertainty of less than 0.9$^\circ$. We use the Fisher distribution
with concentration parameter $\kappa_c = 9322.55$ which corresponds to the
standard deviation of 0.9$^\circ$ in 2-d Gaussian approximation. Let $d_i$ and $\omega_i$
denote the measrued and true directions of UHECR $i$, then
\be
\ell\left(d_i\right) := P\left(d_i|\omega_i\right)={\kappa_c\over4\pi\sinh(\kappa_c)}\exp(\kappa_cd_i\cdot\omega_i)
\ee

\section{Models}
\subsection{The Arrival of UHECRs}
We do not anticipate the UHECR flux passing through a volume element at the Earth to vary in time
over accesible time scales, so we can model the arrival rate into a small volume of space from
any particular direction as a homogeneous Poisson point process in time.
Let $F_k$ denote the UHECR flux from source $k$. $F_k$ is the number of UHECRs per unit time
from source $k$ that would enter a fully exposed spherical detector of unit cross-sectional area.

After leaving their source, UHECRs are expected to undergo deflection due to magnetic field.
Unlike [Imperial], in which the magnetic deflection $\sim 2^\circ$ is used,
we introduce the deflection parameter $\kappa$ as unkown. If AGN$_k$ is the source of UHECR$_i$,
then its direction is assumed to have Fisher distribution centered at the position of AGN$_k$
with concentration parameter $\kappa$:
\be
\rho_k\left(\omega_i|\kappa\right) := P\left(\omega_i|\kappa,\lambda_i=k\right) = {\kappa\over 4\pi\sinh(\kappa)}\exp(\kappa\omega_i\cdot\varpi_k),
\ee
where $\varpi_k$ is the position of AGN$_k$.
When UHECR$_i$ is generated from isotropic background source, the distribution of its
direction is simply the uniform distribution over the sky:
\be
\rho_0\left(\omega_i|\kappa\right) = \rho_0\left(\omega_i\right) = {1\over4\pi}
\ee

Even though the arrival rate of UHECRs into a unit volume is constant in our model, the expected
number per unit time from a given direction will vary as the rotation of the Earth changes
the observatory's projected area toward that direction. Let $A(t_i,\omega_i)$ denote
the observatory's area for detecting UHECR with direction $\omega_i$ arriving at time $t_i$.
Then (?) the projected area of the observatory for detecting UHECR $i$ from source $k$ is
\be
f_{k,i}(\kappa) := \int d\omega_i \ell_i\left(\omega_i\right) A\left(t_i,\omega_i\right) \rho_k(\omega_i|\kappa)
\ee
The effective exposure given to source $k$ throughout the time of the survey $T$ is
\be
\epsilon_k(\kappa) := \int d\omega \rho_k\left(\omega|\kappa\right)\int_Tdt A(t,\omega)
\ee
Let $\lambda_i, i = 1,2,\ldots N_c$, where $N_c$ is the number of UHECRs, be a label specifying
the source of UHECR $i$. That is $\lambda_i = k \geq 1$ if UHECR $i$ comes from AGN $k$ and
$\lambda_i = 0$ if UHECR $i$ comes from the isotropic background source. Using the time-homogeneous
Poisson process, we have the likelihood for $\boldsymbol{F},\kappa$ given as
\be
\ell(\boldsymbol{F},\kappa)=\sum_{\lambda}\left(\prod_k F_k^{m_k(\lambda)}e^{-F_k\epsilon_k}\right) \prod_i f_{\lambda_i,i}
\ee
where $m_k(\lambda)$ is the number of UHECRs assigned to source $k$ accordng to $\lambda$.

Note that since $\kappa_c$ is large, we can approximate $f_{k,i}$ as
\ba
f_{k,i}(\kappa) \approx A_i\cos(\theta_i)\int d\omega_i \ell_i\left(\omega_i\right)\rho_k\left(\omega_i|\kappa\right)
\ea
where $\theta_i$ denotes the zenith angle of UHECR $i$ and $A_i$ is the area of the observatory at the
arrival time of UHECR $i$. The integral can be computed analytically, and is given as
\ba
\int d\omega_i \ell_i\left(\omega_i\right)\rho_k\left(\omega_i|\kappa\right) =
\left\{
\begin{array}{ll}
{\kappa_c\kappa \over 4\pi\sinh(\kappa_c)\sinh(\kappa)}{\sinh(|\kappa_c d_i+\kappa\varpi_k|)\over|\kappa_c d_i+\kappa\varpi_k|} & \mbox{if $k\geq 1$},\\
{1\over 4\pi} & \mbox{if $k=0$}
\end{array}
\right.
\ea
*** computation of $\epsilon_k$ ***

\subsection{Bayesian Hierachical Model}
In this section, we construct a Bayesian hierarchical model for parameters $\kappa,\boldsymbol{F}$ and $\lambda$.
Given the fluxes and $\kappa$, it follows from our Poisson assumption that the probability
that a UHECR comes from a source is proportinal to the expected number of UHECRs from
that source. It should be noted that the number of UHECRs during the time of the survay
is always known. The expected number of UHECRs from a source $k$ is simply $F_k\epsilon_k(\kappa)$.
The prior for $\lambda_i$ is then an independent single-trial multinomial distribution with
distribution function given as
\be
P\left(\lambda_i=k|\boldsymbol{F},\kappa\right) = {F_k\epsilon_k\over \sum F_k\epsilon_k }
\ee

We do our analysis for various values of $\kappa$ between 1 and 1000. To simplify the expresssion,
we will sometimes suppress its notation. We will present the result when $\kappa$ is marginalized
over in section 5.2.

To model the fluxes, we assume that $F_k = w_kF_A$, where $w_k \geq 0$ and $\sum_k w_k = 1$,
for $k\geq 1$, so that our model essentially has 2 unknown flux parameters, the flux from AGN, $F_A$,
and the flux from the isotropic background source, $F_0$. Each specific choice of $w_k$ corresponds
to different assumption of luminosity function of sources.
Let $D$ denote the data. Given the host assignment $\lambda$ and parameter $\boldsymbol{F}$ and $\kappa$,
the likelihood of the data is given as
\be
P\left(D|\lambda,\boldsymbol{F},\kappa\right)=\exp(-\sum F_k\epsilon_k)\left(F_k\epsilon_k\right)^{N_c} \prod_i {f_{\lambda_i,i}\over\epsilon_{\lambda_i}}
\ee
where $N_c$ is the number of UHECRs. Multiplying this by the prior for $\lambda$ and integrating over $\lambda$,
we have that
\be
P\left(D|\boldsymbol{F},\kappa\right) = \sum_{\lambda}\left(\prod_k F_k^{m_k(\lambda)}e^{-F_k\epsilon_k}\right) \prod_i f_{\lambda_i,i}
\ee
which is identical to the likelihood from the UHECR arrival Poisson process.

Plugging in $F_k = w_kF_A$, we can rewrite the previous likelihood as
\ba \label{eq:lik}
P\left(D|F_0,F_A,\kappa\right) &=& \sum_{\lambda} F_0^{m_0(\lambda)}e^{-F_0\epsilon_0} F_A^{N_c-m_0(\lambda)}e^{-F_A\sum w_k\epsilon_k}\nonumber\\
& & \times\prod_{k\geq 1}w_k ^{m_k(\lambda)} \prod_i f_{\lambda_i,i}
\ea
We adopt the conjugate exponential priors with scales $s_0$ and $s_A$ for $F_0$ and $F_A$, respectively.

\subsection{Model Comparison}
Adopting the exponential priors with scales $s_0$ and $s_A$ for $F_0$ and $F_A$, respectively, we have
that the marginal likelihood for $\kappa$ is
\ba  \label{eq:marg}
\ell(\kappa) = P(D|\kappa) &=& \sum_\lambda \left\{{1\over s_0}\left({1\over \epsilon_0+{1\over s_0}}\right)^{m_0(\lambda)+1}\Gamma(m_0(\lambda)+1)\right.\nonumber\\
& & \times {1\over s_A}\left({1\over \sum_{k\geq 1}w_k\epsilon_k+{1\over s_A}}\right)^{N_c-m_0(\lambda)+1}\nonumber\\
& &\left.\times \Gamma\left(N_c-m_0(\lambda)+1\right)\prod_{k\geq1}w_k^{m_k(\lambda)}\prod_i f_{\lambda_i,i}\right\}
\ea
Even though $\ell(\kappa)$ is available in closed form, it requires summing over all possible values of $\lambda$
which is intractable in practice. In later section, we present the Chib estimate for this marginal likelihood.
We use Bayes factors to compare different associations models. The "null" model, M$_0$, assumes that all the
UHECRs come from the isotropic background source. Model M$_1$ allows the UHECRs to come from one of the
17AGNs in the catalog or from the isotropic background. We are also interested in another model M$_2$ in which
the UHECRs comes from the isotropic background or one of the two AGNs, Centaurus A(NGC 5128) and NGC 4945,
 which are the two closest AGNs. In order to compare models M$_1$ and M$_2$ to the null model, we compute the Bayes factors:
\be
\mbox{BF}_{10} = {\ell_1\over\ell_0}, \mbox{ BF}_{20} = {\ell_2\over\ell_0}
\ee
where $\ell_1$ and $\ell_2$ are computed using equation \ref{eq:marg}, and
\be
\ell_0 = {1\over s_0}\left(1\over \epsilon_0+{1\over s_0}\right)^{N_c+1} \Gamma(N_c+1) \times \prod_i f_{0,i}
\ee

\section{Statistical Methods and Algorithms}
\subsection{Prior Specification} In model M$_1$ and M$_2$, we assume that both $F_0$ and $F_A$ have
the exponential prior with the same scale $s$. To keep the prior expected total fluxes the same for the null
model, we use the exponential prior with scale $2s$ for $F_0$ under M$_0$. Using the data from the
two previously operated observatories, AGASA and HiRes, we choose $s\approx 0.063 \mbox{km}^{-1}\mbox{year}^{-1}$

\subsection{Algorithm for Markov Monte Carlo}

To draw posterior inference, we perform Gibbs sampling on parameters $F_0,F_A,\lambda$ with the full conditionals,
derived from  \ref{eq:lik} and the exponential priors, given as
\be
P(F_A|F_0,\lambda,D) \sim \mbox{gamma}\left(\mbox{shape}= 1+\sum_{k\geq 1}m_k(\lambda),\mbox{scale}={1\over{{1\over s} + \sum_{k\geq 1}w_k\epsilon_k}}\right)
\ee
\be
P(F_0|F_A,\lambda,D) \sim \mbox{gamma}\left(\mbox{shape}=1+m_0(\lambda), \mbox{scale}={1\over {1\over s}+\epsilon_0}\right)
\ee
\be
P(\lambda_i=k|F_A,F_0,D) \propto f_{k,i}F_k
\ee


\subsection{Marginal Likelihood and Bayes Factor Computation}
Following Chib(1995), we can write the likelihood for $\kappa$ as
\be
\ell_m = \frac{P\left(D|F_0^*,F_A^*,\lambda^*\right)P\left(F_0^*\right)P\left(F_A^*\right)P\left(\lambda^*|F_0^*,F_A^*\right)}{P\left(F_0^*,F_A^*,\lambda^*|D\right)}
\ee
Here  $F_0^*, F_A^*, \lambda^*$ are chosen from high-posterior points. All the terms in the
 numerator can be computed analytically, using the prios and the likelihood from equation \ref{eq:lik}.
The denominator can be expressed as
\be
P\left(F_0^*,F_A^*,\lambda^*|D\right)=P(F_A^*|F_0^*,\lambda^*,D)P(F_0^*|\lambda^*,D)P(\lambda^*|D)
\ee
The first term on the right hand side is simply the full condition of $F_A$ evaluated at $F_A^*$, $F_0^*$
and $\lambda^*$. Considering the full conditionals given in the previous section, we see that
$\mbox{P}(F_0^*|\lambda^*,D) = \mbox{P}(F_0^*|\lambda^*,F_A^*,D)$, which can also be
computed analytically. The final term $\mbox{P}(\lambda^*|D)$ can be estimated by running additional
$G$ iterations of Gibbs sampling and computing
\be
P(\lambda^*|D) \approx \sum_{g=1}^G P(\lambda^*|F_A^{(g)},F_0^{(g)},D)
\ee
where $F_A^{(g)}$ and $F_0^{(g)}$ are the sampled values of $F_A$ and $F_0$ in iteration $g$, respectively.

\section{Results}
\subsection{Posterior Inference and Bayes Factors}
We do analysis for the cosmic rays from different time peirods and different 
combination of periods. Our main interest is in periods 2\&3 and periods 
1\&2\&3. For each $\kappa$, we first run 50,000 iterations of Gibbs 
sampling to obtain high posterior points $F_0^*, F_A^*, \lambda^*$ 
to be used in Chib estimate. Then, for each $\kappa$, 3 additional 
chains of Gibbs sampling are run, each with 50,000 iterations. 
The chains are thinned so that the lag-1 autocorrelation is at 
most 0.15 for all parameters $F_0, F_A, \lambda$. The Bayes 
factors comparing models M$_1$ and M$_2$ to M$_0$ for
various values of $\kappa\in[1,1000]$ in different periods are 
shown in Fig.\ref{fig:BFplot}. Considering only the cosmic rays 
from periods 2$\&3$, we find that both BF$_{10}$ and BF$_{20}$
are close to 1 for all values of $\kappa\in[1,1000]$. There is no 
conclusive evidence when comparing the association models 
M$_1$ and M$_2$ to the null isotropic background model M$_0$.
When considering the cosmic rays from all the periods 1$\&$2$\&$3, 
we obtain different results. BF$_{10}$ attains its maximum 
of 84 at $\kappa\approx 46$ while BF$_{20}$ attains its maximum of 
240 at $\kappa\approx 38$. Both BF$_{10}$ and BF$_{20}$ are 
larger than 30 when $\kappa\in[20,120]$. Both of the association 
models are preferred over the null in this range of $\kappa$, 
while the comparison is inconclusive for $\kappa$ outside this range.

The joint posterior distributions for $(F_0,F_A)$ for some values 
of $\kappa$ are shown in Fig.\ref{fig:jointF}. The summary from 
the posterior distribution of $f$ is shown in Table\ref{tab:sumpostf}.

\begin{table}[h]
\begin{tabular}{|c|c |c| c| c| c|c|c|c|}
\hline
& \multicolumn{4}{|c|}{17 AGNs} & \multicolumn{4}{|c|}{2 AGNs}\\
\cline{2-9}
& \multicolumn{2}{|c|}{periods 2\&3} & \multicolumn{2}{|c|}{periods 1\&2\&3}& \multicolumn{2}{|c|}{periods 2\&3} & \multicolumn{2}{|c|}{periods 1\&2\&3}\\
\cline{2-9}
$\kappa$&mode&{95\% CI} & mode&{95\% CI}& mode&{95\%CI} & mode&{95\% CI}\\
\hline
10 & 0.12  & [0.00,0.27] & 0.14 &[0.04,0.30] & 0.07 & [0.00,0.17] & 0.10 &[0.03,0.20]\\
31.62 & 0.09 & [0.01,0.20] & 0.13 &[0.05,0.24] & 0.06 &[0.01,0.14] & 0.09 & [0.03,0.16]\\
100 & 0.05 & [0.00,0.14] & 0.09 & [0.03,0.19] & 0.04 & [0.00,0.10] & 0.06 & [0.02,0.12]\\
316.2 & 0.04 & [0.00,0.11] & 0.06 & [0.01,0.14] & 0.02 & [0.00,0.07] & 0.03 & [0.00,0.08]\\
1000 & 0.03 & [0.00,0.10] & 0.04 & [0.00,0.10] & 0.02 & [0.00,0.06] & 0.02 & [0.00,0.06]\\
\hline
\end{tabular}
\caption{Posterior mode and 95\% highest posterior density credible interval for $f$}\label{tab:sumpostf}
\end{table}

\begin{figure}
\centerline{$
\begin{array}{cc}
\includegraphics[width=2in,angle=-90]{BF_plot_vs_null_17AGNs.eps} &
\includegraphics[width=2in,angle=-90]{BF_plot_vs_null_2AGNs.eps}
\end{array}$}
\caption{Bayes factors comparing the assocation model with 17 AGNs (left) or 2 AGNs(right) with the null
isotropic backgroud model}
\label{fig:BFplot}
\end{figure}

\subsection{Results from Marginalizing over $\kappa$}
We adopt a log-flat prior for $\kappa\in[1,1000]$. That is
\be
P(\kappa) = {1\over\kappa\log(1000)}, \mbox{ for } 1\leq\kappa\leq1000
\ee
The posterior joint distributions of $(\kappa,f)$ are shown 
in Fig.\ref{fig:jointkappaf}. Since the data is noninformative when 
we consider only the cosmic rays from periods 2\&3 (Bayes factor stay 
close to 1 for all $\kappa\in[1,1000]$), the data and the prior put equally 
weights on the joint posteriors. The posteriors in this case are bimodal with one 
local mode at $\kappa$ close to 1, similar to the prior, and the other close to
the mode of the marginal likelihood for $\kappa$ (which is proportional 
to the Bayes factors given in the previous section).  When the cosmic rays 
from all the periods are considered, the joint poterior distribution
for ($\kappa,f)$ is unimodal and attains its maximum at 
($\kappa$=32 , $f$=0.13) and ($\kappa$=32 , $f$=0.09) 
for the association model with 17 AGNs and 2 AGNs, respectively.

\begin{figure}
\centerline{$
\begin{array}{cc}
\includegraphics[width=2in,angle=-90]{posteriorJoint_fkappa_17AGNs_periods23.eps} &
\includegraphics[width=2in,angle=-90]{posteriorJoint_fkappa_17AGNs_periods123.eps}\\
\includegraphics[width=2in,angle=-90]{posteriorJoint_fkappa_2AGNs_periods23.eps} &
\includegraphics[width=2in,angle=-90]{posteriorJoint_fkappa_2AGNs_periods123.eps}
\end{array}$}
\caption{Joint posterior distributions for ($\kappa,f$)}
\label{fig:jointkappaf}
\end{figure}

The posterior distributions of $f$ are shown in Fig.\ref{fig:postf2}. For the cosmic rays in periods 2\&3,
the posterior mode of $f$ is 0.05 for both of the assocation models with 17 AGNs and 2 AGNs. For the
cosmic rays in periods 1\&2\&3, the posterior mode of $f$ is 0.10 and 0.08 for the association models with
17 AGNs and 2 AGNs, respectively.
\begin{figure}
\centerline{\includegraphics[width=3in,angle=-90]{posterior_f_all_margOverKappa.eps} }
\caption{Posterior distributions for $f$ after marginalizing over $\kappa$}
\label{fig:postf2}
\end{figure}

The overall Bayes factors after marginalizing over $\kappa$ are shown in Table\ref{tab:BFtab}. We find strong
evidence of association only in the model with 2 AGNs when considering the cosmic rays from periods 1\&2\&3
with the Bayes factor of 46.76.
\begin{table}
\begin{tabular}{|c|c |c| c| c| c|}
\hline
& \multicolumn{5}{|c|}{Bayes factors}\\
\cline{2-6}
 &Period 1 & Period 2 & Period 3 & Periods 2\&3 & Periods 1\&2\&3\\
\hline
17 AGNs & 26.10 & 5.41 & 0.15 & 0.94 & 24.10\\
2 AGNs  & 12.37 & 8.27 & 0.11 & 1.01 & 46.76\\
\hline
\end{tabular}
\caption{Overall Bayes factors comparing association models with 17 AGNs (top row) 
or 2 AGNs (bottom row) to the null isotropic background model}\label{tab:BFtab}
\end{table}


\subsection{Investigating the Difference between Time Periods}
\section{Simulation}
\section{Conclusion}




% AOS,AOAS: If there are supplements please fill:
%\begin{supplement}[id=suppA]
%  \sname{Supplement A}
%  \stitle{Title}
%  \slink[url]{http://lib.stat.cmu.edu/aoas/???/???}
%  \sdescription{Some text}
%\end{supplement}


\end{document}
