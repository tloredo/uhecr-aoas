\section{Simulation study}\label{sec:simulations}

As seen in Section 5.2, adding the CR data from period 1 markedly changes
the Bayes factors; yet, formal hypothesis testing in Section 5.3 prefers the models
with the same parameters in all periods. In this section, we use simulations to  study the potential random variation of the Bayes factors among the 3 periods. The goal is to
answer the following two questions: 1) under the null isotropic background model,
how likely it is to see Bayes factors as high as 100  for some values of $\kappa$ in period 1 (as seen in Figure~\ref{fig:BFplot}),
and 2) under the association model with 17 AGNs, how likely it is to see large Bayes factors
in periods 1 or 2 (which have small number of detected CRs) and small Bayes factors in
period 3 (which has higher number of detected CRs).

To answer 1), we generated 200 datasets, each of which has 14 CRs. The CR directions are
generated uniformly over the sky, and are accepted with the probability proportional
to the exposure map for that direction. We performed Gibbs sampling at $\kappa
= 10, 31.62, 100, 316.2$ and 1000. Figure~\ref{fig:unifCumBF} shows the cumulative histogram (as a survival function) of the
Bayes factors of these 200 datasets for each value of $\kappa$.
Most of the datasets have Bayes factors less than 1. The smallest Bayes factor
we found is 0.04. Out of these 200 datasets, we found only 3 datasets
that have $B_{10}\geq10$. Each of these datasets has $B_{10}\geq10$ only at one value
of $\kappa$ that we computed. The Bayes factors are 29 ($\kappa=10$), 44 ($\kappa = 31.62$)
and 13 ($\kappa = 316.2$).
We conclude that the Bayes factor greater than 100 seen in period 1 is unlikely to be due to chance, although it could, of course, be due to tuning.

\begin{figure}
\centerline{\includegraphics[width=3in,angle=-90]{BF_cumplot_genUnif_14CRs_logscale.eps} }
\caption{The number of datasets having Bayes factor $\geq B$ vs. $B$. Each of the 200 datasets has
14 CRs generated under the null isotropic background model.}
\label{fig:unifCumBF}
\end{figure}

To answer 2), we generated 100 datasets, each of which has 69 CRs, 14, 13 and 42 for periods
1, 2 and 3, respectively. Each CR has a probability of $f=0.1$ of being generated from an AGN,
and $1-f = 0.9$ of being generated from the background source. AGN-generated CRs get
assigned to one of 17 AGNs with probabilities proportional to the inverse square distance
of the AGNs. The arrival directions of these CRs follow a Fisher distribution with $\kappa = 50$ and
centered at its AGN source. All the generated CR directions are accepted with probability
proportional to the exposure map for that direction. Out of these 100 datasets, we found
17 of them that have some of the Bayes factors $\geq 30$.  Of these, 9 datasets have
some values of $B_{10}\geq 30$ in periods 1 or 2 and low $B_{10}$ values in period 3, while the other
8 datasets have some values of $B_{10}\geq 30$ in period 3
and low $B_{10}$ values in periods 1 and 2. Out of these 17 datasets, 10 of them have some values
of $B_{10}\geq 100$. These Bayes factors are shown in Table~\ref{tab:BFsim}.
We conclude that Bayes factor as high as 100 (as seen in period 1) are likely under the association model and the large between-period variation seen in the data are likely.



\begin{table}[h]
\begin{tabular}{|c|c |c| c| c| c|c|c|c|c|c|c|}
\hline
 & & \multicolumn{10}{|c|}{Dataset}\\
\cline{3-12}
Period & $\kappa$ & 19 & 48 & 51 & 57 & 69 & 73 & 84 & 89 & 93 & 100\\
\hline
1 & 10 & 0.50 & 0.26 & 15 & 1.44 & 0.16 & 1.27 & 172 & 0.12 & 1.38 & 0.28\\
   & 32 & 0.17 & 0.28 & 126 & 2.72 & 0.14 & 1.98 & 1198 & 0.14 & 0.45 & 0.58\\
   & 100 & 0.09& 0.33 & 202 & 3.34 & 0.14 & 0.83 & 1240 & 0.15 & 0.27 & 1.06\\
   & 316 & 0.05 & 0.29 & 7.11 & 1.41 & 0.13 & 0.08 & 160 & 0.09 & 0.27 & 0.38\\
   & 1000 & 0.04 & 0.15 & 0.20 & 0.23 & 0.07& 0.06 &  1.60 & 0.05 & 0.16 & 0.11\\
\hline
2 & 10 & 0.22 & 49 & 0.14 & 0.09 & 4.06 & 0.12 & 0.24 & 0.39 & 6.04 & 0.11\\
   & 32 & 0.13 & 221 & 0.14 & 0.06 & 61 & 0.10 & 0.14 & 0.28 & 27 & 0.09\\
   & 100 & 0.06 & 557 & 0.13 & 0.04 & 767 & 0.10 & 0.09 & 0.19 & 124 & 0.07\\
   & 316 & 0.05 & 71 & 0.06 & 0.04 & 1742 & 0.07 & 0.09 & 0.09 & 218 & 0.05\\
   & 1000& 0.04 & 0.88 & 0.04 & 0.04 & 41 & 0.04 & 0.07 & 0.05 & 193 & 0.04\\
\hline
3 & 10 & 32 & 0.11 & 0.07 & 10.38 & 0.28 & 93 & 0.10 & 19 & 0.53 & 59\\
   & 32 & 174 & 0.17 & 0.09 & 175 & 0.14 & 146 & 0.04 & 130 & 0.68 & 539\\
   & 100 & 7.35 & 0.28 & 0.19 & 396 & 0.04 & 30 & 0.03 & 261 & 0.49 & 119\\
   & 316 & 0.18 & 0.36 & 0.23 & 25 & 0.02 & 0.87 & 0.02 & 132 & 0.26 & 0.87\\
   & 1000 & 0.03 & 0.29 & 0.17 & 1.63 & 0.02 & 0.31 & 0.02 & 1.26 & 0.08 & 0.03\\
\hline
\end{tabular}
\caption{Bayes factors from the simulated dataset from 17-AGN association model}\label{tab:BFsim}
\end{table}
