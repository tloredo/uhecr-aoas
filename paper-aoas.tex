\documentclass[dvips,aoas,preprint]{imsart}
%\documentclass[dvips,aoas]{imsart}

\RequirePackage[OT1]{fontenc}
\RequirePackage{amsthm,amsmath}
\RequirePackage[numbers]{natbib}
\RequirePackage[colorlinks,citecolor=blue,urlcolor=blue]{hyperref}

\usepackage{amssymb,amsbsy}
\usepackage{mathtools}  % for \coloneqq
%\usepackage[dvips]{graphics}
\usepackage{graphicx}
\usepackage{rotating}  % for sidewaystable

% Packages that aren't standard for AOAS:
%\usepackage{subfig} 

% Watermarking:
%\usepackage{draftwatermark}
%\SetWatermarkText{DRAFT}
%\SetWatermarkLightness{0.9}
%\SetWatermarkScale{5}  % default is 1.2

% To typeset latex command names:
\newcommand\cmdname[1]{{\tt \textbackslash#1}}
\chardef\bslash=`\_

%\usepackage{amsthm,amsmath,natbib}
%\RequirePackage[colorlinks,citecolor=blue,urlcolor=blue]{hyperref}

% provide arXiv number if available:
%\arxiv{math.PR/0000000}

%===============================================================================
% put your definitions there:
\startlocaldefs

% Macro for an "editorial note"; uncomment the 2nd definition to
% produce a version without notes (put it in main doc if desired).




\newcommand\enote[1]{{$\bullet\bullet\bullet$}{\sl [#1]}{$\bullet\bullet\bullet$}}

%\renewcommand\enote[1]{\relax}





% Marginal note (just a redef of \marginpar or \marginparodd).
% Uncomment the 2nd definition to produce a version without notes (put it
% in main doc if desired).
% Could use \footnotesize for a bigger font.
% The \hspace enables hyphenation of the 1st word.
\setlength{\marginparwidth}{.85in}
\newcommand\mnote[1]{\-\marginpar[\raggedleft\scriptsize\hspace*{0pt}#1]%
{\raggedright\scriptsize\hspace*{0pt}#1}}
%\renewcommand\mnote[1]{\relax}

\newcommand{\txw}{\textwidth}

\renewcommand{\arraystretch}{1.2}


\newcommand{\bm}[1]{\mbox{\boldmath{$#1$}}}

\newcommand{\be}{\begin{equation}}
\newcommand{\ee}{\end{equation}}
\newcommand{\ba}{\begin{eqnarray}}
\newcommand{\ea}{\end{eqnarray}}
\newcommand{\barr}{\begin{array}}
\newcommand{\earr}{\end{array}}
\newcommand{\bc}{\begin{center}}
\newcommand{\ec}{\end{center}}

\newcommand{\like}{{\cal L}}
\newcommand{\pval}{$p$-value}
\newcommand{\Eth}{E_{\rm th}}
\newcommand{\Dmax}{D_{\rm max}}
\newcommand{\Nsrc}{N_S}

% Coincidence model macros; t* = target, h* = host:
\newcommand{\tdrxn}{\omega}  % target (CR) direction
\newcommand{\hdrxn}{\varpi}  % host (AGN) direction
\newcommand{\expo}{\epsilon}  % exposure
\newcommand{\tlabel}{\lambda}  % target label
\newcommand{\Fvec}{\bm{F}}  % host fluxes

\newcommand{\D}{\overline{\mbox{D}}}

\newcommand{\simless}{\mathbin{\lower 3pt\hbox
   {$\rlap{\raise 5pt\hbox{$\char'074$}}\mathchar"7218$}}}
\newcommand{\simgreat}{\mathbin{\lower 3pt\hbox
   {$\rlap{\raise 5pt\hbox{$\char'076$}}\mathchar"7218$}}}
\newcommand{\gta}{\simgreat}
\newcommand{\lta}{\simless}
\newcommand\msun{{\rm M}_\odot}
\endlocaldefs

%===============================================================================
\begin{document}

%===============================================================================
\begin{frontmatter}

% "Title of the paper"
\title{Multilevel Bayesian Framework for Modeling the Production,
Propagation and Detection of Ultra High Energy Cosmic Rays}

% For footnote indicating draft status:
% END OF TITLE TEXT\protect\thanksref{DRAFT}}
%\thankstext{DRAFT}{This is an description of work in progress, prepared for
%participants in the 2011 workshop, {\em Case Studies in Bayesian Statistics and
%Machine Learning}.  {\bf PLEASE DO NOT CIRCULATE!}}

% indicate corresponding author with \corref{}
% \author{\fnms{John} \snm{Smith}\corref{}\ead[label=e1]{smith@foo.com}\thanksref{t1}}
% \thankstext{t1}{Thanks to somebody}
% \address{line 1\\ line 2\\ printead{e1}}
% \affiliation{Some University}

\author{\fnms{Kunlaya} \snm{Soiaporn}\corref{}\ead[label=e1]{ks354@cornell.edu}},
\address{\printead{e1}}
\author{\fnms{Thomas} \snm{Loredo}\ead[label=e2]{loredo@astro.cornell.edu}},
\address{\printead{e2}}
\author{\fnms{David} \snm{Chernoff}\ead[label=e3]{cheroff@astro.cornell.edu}},
\address{\printead{e3}}
\author{\fnms{David} \snm{Ruppert}\ead[label=e4]{dr24@cornell.edu}}
\address{\printead{e4}}
\and
\author{\fnms{Ira} \snm{Wasserman}\ead[label=e5]{ira@astro.cornell.edu}}
\address{\printead{e5}}
\affiliation{Cornell University}


\runauthor{Soiaporn, et al.}
\runtitle{Multilevel Models for Cosmic Rays}

\begin{abstract}
Cosmic rays are atomic nuclei with velocities close to the speed of light.
Ultra-high energy cosmic rays (UHECRs) are approximately ten million times
more energetic than the most extreme particles produced at the Large Hadron
Collider and likely originate from nearby extragalactic sources. Important
astrophysical questions include: what phenomenon accelerates particles to such
large energies, which astronomical objects host the accelerators, and what
sorts of nuclei end up being energized? We develop a multilevel Bayesian
framework for assessing evidence for association of UHECRs and candidate source
populations that has four levels:
(1) a source property description level, including specification of a
candidate source population and the distribution of cosmic ray
intensities among sources;
(2) a cosmic ray production level, using a marked Poisson point process
to model latent properties (arrival times and energies) of cosmic rays;
(3) a propagation level, modeling deflection of cosmic ray trajectories by
cosmic magnetic fields;
(4) a detection and measurement level, accounting for detector efficiency
and exposure, and measurement errors in arrival directions and energies.
Our framework can flexibly accomodate astrophysical components anywhere on
a spectrum from simplicity to great complexity and realism.  
We demonstrate the framework using simple models similar to those
adopted in previous studies, but our results suggest more complex models
are likely necessary (we will pursue such models in future work).

The principal obstacle to computing with this framework is the
combinatorial explosion in the number of possible associations as the
sizes of the candidate source population and the UHECR sample grow.  We
present a Markov chain Monte Carlo algorithm for estimating the
parameters and computing, via Chib's method, the marginal likelihoods
used in model comparison.  We apply the algorithm to the observations of
69 UHECRs observed by Pierre Auger Observatory (PAO) during the years
2004--2009, using a volume-complete catalog of 17 nearby active galactic
nuclei as a candidate host population.
The reported data are incomplete; an early portion of the data
(``Period~1'') was used to set an energy cut maximizing a measure of
anisotropy in that data; only data for cosmic rays with energies above
that cut is reported, in Period~1 (14 events) and subsequently.  Also,
measurement errors are approximately summarized.  These factors are
problematic for independent analyses of PAO data.  Within the context of
``standard candle'' source models (all sources with the same isotropic
cosmic ray emission rate), there is no significant evidence favoring
association of UHECRs with nearby AGN vs.\ attributing them to sources
drawn from an isotropic background distribution, when we analyze the
untuned data subsequent to Period~1; the highest-probability
associations are with the two nearest AGN, Centaurus~A and its neighbor,
NGC~4945.  If the association model is adopted, the fraction of UHECRs
that may be associated is likely nonzero but is constrained to be well
below 50\%.  Relatively small magnetic deflection angular scales of
$\approx 3^\circ$ to $20^\circ$ are favored; models that assign a large
fraction of UHECRs to a single nearby source are ruled out unless very
large deflection scales are specified a priori, and even then they are
disfavored.  However, including the Period~1 data alters the conclusions
significantly, and a simulation study supports the idea that the
Period~1 data are anomalous, presumably due to the tuning.  Accurate and
optimal analysis of future data will likely require more complete
disclosure of the data.
\end{abstract}


%\begin{keyword}[class=AMS]
%\kwd[Primary ]{}
%\kwd{}
%\kwd[; secondary ]{}
%\end{keyword}

\begin{keyword}
\kwd{Multilevel modeling}
\kwd{Hierarchical Bayes}
\kwd{Astrostatistics}
\kwd{Cosmic rays}
\kwd{Directional data}
\kwd{Coincidence assessment}
\kwd{Bayes factors}
\end{keyword}

\end{frontmatter}

%===============================================================================

%\enote{Editorial notes appear like this, using the \cmdname{enote} macro
%defined in paper.tex.  For issues not resolved but being deferred, leave the
%\cmdname{enote} in place, but uncomment the \cmdname{renewcommand}
%redefinition of \cmdname{enote} in paper.tex to hide them; we'll deal with
%them later.}


\input{intro-aoas}

\section{Description of cosmic ray and candidate host data}

%..............................................................................
\subsection{Cosmic ray data}
\label{sec:data}

The reported PAO measurements depend not only on the intrinsic particle
population but also on many experimental and algorithmic choices in the
detection and analysis chain, many of them associated with the need to
distinguish between events of interest and background events from uninteresting
but uncontrollable sources (e.g., natural radioactivity).  UHECRs can
impinge on the observatory at any time, from any direction and with any
energy.
However, virtually no background sources
produce events with properties mimicking those of very high energy cosmic rays
arriving from directions well above the horizon.
Cosmic rays with $E>3$~EeV arriving from any direction lying within a large
window on the sky create air showers detected with nearly 100\% efficiency
(no false positives, no false dismissals).  The SDs and FDs measure
the spatio-temporal development of the air shower which allows
the energy and arrival direction to be measured.  The uncertainties depend
upon how many counters of each type are triggered plus the systematic and
statistical uncertainties implicit in modeling  the development of the
air shower. The PAO team reports energy and
arrival direction estimates for each cosmic ray falling within the geometric
bounds of its zone of secure detection.\footnote{The directional criterion
adopted for the PAO catalogs is that an event is reported if it best-fit
arrival direction is within $60^\circ$ of the observatory's zenith, the
local normal to Earth's surface at the time of the event.}

We consider the $N_C = 69$ UHECRs with energies $E \geq \Eth = 55$~EeV
cataloged in PAO-10, which reports measurements of all UHECRs seen by PAO
through 31 December 2009 with $E \geq \Eth$, based on analysis of the
surface detector data only.  Although our framework does not tune an event
selection criterion, for interpreting the results it is important to
remember that the $\Eth=55$~EeV threshold value was set to maximize a
signature of anisotropy in an early subset of the data.
The tuning data included the 14 earliest
reported events, detected from 1~January 2004 to 26~May 2006 (inclusive;
Period~1), as well as numerous unreported events with $E<\Eth$.  The first
published catalog in PAO-08 included 13 subsequent UHECRs observed through 31
August 2007 (Period~2).  The PAO-10 catalog includes 42 additional UHECRs
observed through 31 December 2009 (Period~3). Table~1 in PAO-10 provides
information about the three periods, including the sky exposure for each
period, which is not simply proportional to duration (the observatory grew in
size considerably through 2008).  Data for cosmic rays with $E < \Eth$ are not
publicly available.\footnote{The PAO web site hosts public data for 1\% of
lower-energy cosmic rays, but the sample is not statistically characterized
and UHECRs are not included.}

The direction estimate for a particular cosmic ray is the result of a
complicated analysis of time series data from the array of PAO surface
detectors.\footnote{The SD data may be supplemented by data from the
fluorescence detectors for hybrid events observed under favorable
conditions, but there are very few such events at ultra-high energies,
and PAO-10 reports analysis of SD data only.}
Roughly speaking, the direction is inferred by triangulation. 
The analysis produces a likelihood function for the cosmic ray arrival
direction, $\omega$ (a unit vector on the celestial sphere).  The shapes of
the likelihood contours are not simple, but they are roughly azimuthally
symmetric about the best-fit direction.  The PAO-10 catalog summarizes the
likelihood function with a best-fit direction, and a typical directional
uncertainty of $\approx 0.9^\circ$ corresponding to the angular radius of an
azimuthally symmetric 68.3\% confidence region.  We use these summaries to
approximate the likelihood functions with a Fisher distribution with mode at
the best-fit direction for each cosmic ray, and with concentration parameter
$\kappa_c = 9323$, corresponding to a 68.3\% confidence region with an
angular radius of 0.9$^\circ$.  Let $d_i$ denote the data associated with
cosmic ray $i$, and $\omega_i$ denote its actual arrival direction.  The
likelihood function for the direction is
\ba
\ell(\omega_i)
  \coloneqq P(d_i|\omega_i)
   \approx \frac{\kappa_c}{4\pi\sinh(\kappa_c)} \exp(\kappa_c n_i\cdot\omega_i),
\label{ell-def}
\ea
where $n_i$ denotes the best-fit direction for cosmic ray $i$, and we have
scaled the likelihood function so its integral over $\omega_i$ is unity,
merely as a convenient convention.  Bonifazi et al.\
\cite{B+PAO09-DrxnUncert} provide more information about the PAO direction
measurement capability.  Note that the expected angular scale of magnetic
deflection is larger than the PAO directional uncertainties, significantly
so if UHECRs are heavy nuclei (see \S~\ref{sec:dflxn}).

Similarly, the analysis pipeline produces energy estimates for each event.
These estimates have significant random and systematic uncertainties
\cite{PAO08-GZK,PAO10-GZK}.
The models we study here do not make use of the reported energies and are
unaffected by these uncertainties.  But our framework readily generalizes to
account for energy dependence.  In principle it is straightforward to account
for the random uncertainties, but a consistent treatment requires data for
events below any imposed threshold:  the true energies of events with best-fit
energies below threshold could be above threshold (and vice versa for those
with best-fit energies above threshold); accounting for this requires data to
energies below astrophysically important thresholds.  The systematic
uncertainties become important for joint analyses of PAO data with data from
other experiments, and for linking results of spectral analyses to particle
physics theory.


%..............................................................................
\subsection{Candidate source catalog}

As candidate sources for the PAO UHECRs, the analysis reported in PAO-07 and
PAO-08, and several subsequent analyses, considered 694 
AGN within $\approx 75$~Mpc from the 12th catalog 
assembled by V\'eron-Cetty and V\'eron \cite{VCV-12thAGNCat} (VCV). 
This catalog includes data on all AGN and quasars (AGN with star-like
images) with published spectroscopic redshifts; it includes observations
from numerous investigators using diverse equipment and AGN selection
methods, and does not represent a statistically well-characterized sample of
AGN.\footnote{VCV say of the catalog, ``This catalogue should not be used
for any statistical analysis as it is not complete in any sense, except that
it is, we hope, a complete survey of the literature.''}
Subsequent analyses in PAO-10, and a few other analyses, used more recent
catalogs of active galaxies or normal galaxies, including flux-limited
catalogs (i.e., well-characterized catalogs that contain all bright sources
within a specified volume, but dimmer sources only in progressively smaller
volumes).

%As noted above, the Greisen-Zatsepin-Kuzmin (GZK) limit implies that cosmic
%rays with energies $\gtrsim$ 50~Eev should interact with cosmic microwave
%background photons and should almost never reach the earth from distance
%longer than 50~Mpc.

For the representative analyses reported here, we consider the 17 AGN
cataloged by Goulding et al.\cite{2010MNRAS.406..597G} (2010; hereafter G10)
as candidate sources.  This is a well-characterized {\em volume}-limited
sample; it includes all infrared-bright AGN within 15~Mpc.  
For each AGN in the calalog, we take its position on the sky, $\varpi_k$
($k=1$ to $\Nsrc$), and its distance, $D_k$, to be known precisely (galaxy
directions have negligible uncertainties compared to cosmic ray directions).
Notably, this catalog includes Centaurus~A (Cen~A), the nearest AGN
($D\approx 4.0$~Mpc), an unusually active and morphologically peculiar AGN.
Theorists have hypothesized Cen~A to be a source of many or even most UHECRs
if UHECRs are heavy nuclei, which would be deflected through large angles;
see \cite{B+09-CenA,GBdS10-CenA,BdS12-CenA}.  The small size of this catalog
facilitates thorough exploration of our methodology: Markov chain Monte
Carlo algorithms can be validated against more straightforward algorithms
that could not be deployed on large catalogs, and simulation studies are
feasible that would be too computationally expensive with large catalogs.
Also, for simple ``standard candle'' models (adopted here and in other
studies), that assign all sources the same cosmic ray intensity, little is
gained by considering large catalogs, because assigning detectable cosmic
ray intensities to distant sources would imply cosmic ray fluxes from nearby
sources too large to be compatible with the data.

We also include an isotropic background component as a ``zeroth" source.
This allows a model to assign some UHECRs to sources not included in the AGN
catalog (either galaxies not cataloged, or other, unobserved sources). In
addition, we consider an isotropic source distribution for {\em all} cosmic
rays (i.e., a model with only the zeroth source) as a ``null'' model for
comparison with models that associate some cosmic rays with AGN or other
discrete sources.  An isotropic distribution is convenient for calculations
and has been adopted as a null hypothesis in several previous studies. 
Historically, before PAO's convincing observation of a GZK-like cutoff in
the UHECR energy spectrum, the isotropic distribution was meant to represent
a distant cosmological origin for UHECRs.  Accepting the null would indicate
that the GZK prediction was incorrect, and that changes in fundamental
physics would be required to explain UHECRs.  In light of PAO's compelling
observation of a GZK-like cutoff (with its implied $\sim 100$~Mpc distance
scale), interpreting an isotropic null or background component is
problematical if there are many light nuclei among the UHECRs.  We adopt it
here both for convenience and due to precedent.  We discuss this further
below.

%..............................................................................
\subsection{Sky map}

Figure~\ref{fig:skymap} shows a sky map displaying the
directions to both the UHECRs seen by PAO, and the AGN in the G10 catalog.
The directions are shown in an equal-area Hammer-Aitoff projection in
Galactic coordinates; the Galactic plane is the equator (Galactic latitude
$b=0^\circ$), and the vertically-oriented grid lines are meridians of
constant Galactic longitude, $l$.  The star indicates the south celestial
pole (SCP), the direction directly above Earth's south pole (effects like
precession and nutation of the Earth's axis are negligible for this
application and we ignore them in this description).  The thick gray line
bounds the PAO field of view.  The UHECR and AGN directions are displayed as
``tissots,'' projections of circular patches centered on the reported
directions.  The small tissots show the UHECR directions; the tissot size is
$2^\circ$, corresponding to $\approx 2$ standard deviation errors, and the
tissot color indicates energy.  The large green tissots indicate AGN
directions; the tissot size is $5^\circ$, corresponding to a plausible scale
for magnetic deflection of UHE protons in the Galactic magnetic
field.\footnote{If UHECRs are comprised of heavier, more positively charged
nuclei, they could suffer much larger deflections; see \S~\ref{sec:dflxn}.}
The tissots are rendered with transparency; the two darker tissots near the
Galactic north pole indicate pairs of AGN with nearly coincident directions.
Two of the AGN tissots are outlined in solid black; these correspond to the
two nearest AGN, Centaurus~A (Cen~A, also known as NGC~5128) and NGC~4945,
neighboring AGN at distances of 4.0 and 3.9~Mpc (as reported in G10).  Five
others are outlined in dashed black; these have distances ranging from 6.6
to 10.0~Mpc. The remaining 10 AGN have distances from 11.5 to 15.0~Mpc. 
Four of the AGN are outside the PAO field of view, but depending on the
scale of magnetic deflection, they could be sources of observable cosmic
rays.


\begin{figure}
\begin{centering}
\includegraphics[width=\textwidth]{CR+LocalAGN-all.eps}
\end{centering}
\caption{Sky map showing directions to 69 UHECRs detected by PAO, and
to 17 nearby AGN from the catalog of Goulding et al..  Directions are
shown in an equal-area Hammer-Aitoff projection in Galactic coordinates.
Thick gray line indicates the boundary of the PAO field of view.
Small tissots show UHECR directions; tissot radius is $2^\circ$ corresponding
to $\approx 2$ standard deviation errors; tissot color indicates energy.
Large green tissots indicate AGN directions; tissot radius is $5^\circ$.
Thin curves are geodesics connecting each UHECR to its nearest AGN.}
\label{fig:skymap}
\end{figure}

Figure~\ref{fig:skymap} shows the measured directions for the 69 UHECRs.
The thin curves (teal) show
geodesics connecting each UHECR to its nearest AGN.
There is a noticable concentration of cosmic ray directions near the
directions of Cen~A and NGC~4945; a few other AGN also have conspicuously
close cosmic rays.  We have also examined similar maps for the subsets
of the UHECRs in the three periods.
The concentration in the vicinity of the two closest AGN
is also evident in the maps for periods~1 and 2.
Curiously, except for a single UHECR about $6^\circ$
from NGC~4945, no such concentration is evident in the map for Period~3,
despite it having about three times the number of UHECRs found in earlier
periods.  This is a presage of results from our quantitative analysis that
suggest the data may not be consistent with simple models for the cosmic ray
directions, with or without AGN associations.

%..............................................................................
\subsection{PAO exposure}
\label{sec:expo}

PAO is not equally sensitive to cosmic rays coming from all directions.
Quantitative assessment of evidence for associations or other anisotropy must
account for the observatory's direction-dependent exposure.

Let $F$ be the cosmic ray flux at Earth from a source at a given direction,
$\tdrxn$, i.e., the expected number of cosmic rays per unit time per unit
area normal to $\tdrxn$.   Then the expected number of rays detected in a
short time interval $dt$ is $F \Aperp(t,\tdrxn) dt$, where $\Aperp(t,\tdrxn)$ is
the projected area of the observatory toward $\tdrxn$ at time $t$. The total
expected number of cosmic rays is given by integrating over $t$; it can be
written as $F\expo(\tdrxn)$, with the {\em exposure map} $\expo(\tdrxn)$
defined by
\begin{equation}
\expo(\tdrxn) \coloneqq \int_T  \Aperp(\tdrxn,t) dt;
\label{expo-def-main}
\end{equation}
the integral is over the time intervals when the observatory was operating,
denoted collectively by $T$.
The \ref{supp} \cite{S+12-UHECR-Supp} describes calculation of
$\expo(\tdrxn)$; the thick gray curves shown in the sky maps mark the
boundary of the region of nonzero exposure.
  % omits extra skymaps

\input{modeling}

\input{results}

% This is in place of the model checking section, which appears here in
% the arXiv version:


\subsection{Model checking}

\mnote{New subsection; model checking section moved to appendix.}
In Online Appendix~\ref{app:checking} we describe results of two types of
tests of our models, motivated by period-to-period variability of some of
the results reported above.

First, we performed simple change point analyses to see whether the
period-to-period variation of the Bayes factors for association vs.\ isotropy
indicates the population-level properties of the detected cosmic rays vary
from period to period.  We compared versions of $M_1$ and $M_2$ that allow
model parameters to change between periods to versions that keep the
parameters the same for all periods.  We find that there is no significant
evidence for variability of model parameters from period to period.  

%That is, presuming one of the models is adequate, the apparent discrepancy
%among the Bayes factors in Table~\ref{BFTable} reflects variability that may
%be expected for these modest sample sizes.

Second, we performed predictive checks to see whether the period-to-period
Bayes factor variations are surprising in the context of either the null or
association models, essentially using the Bayes factors as goodness-of-fit
test statistics.  
We simulated data from the null (isotropic) model and compared the Bayes
factors based on the observed data with those found in the simulations; we
did the same for a representative association model.
We find that Bayes factors favoring association as large as that
found with the Period~1 PAO data are unlikely for
isotropic models.
This implies the distribution of directions in the Period~1 sample is
anisotropic, but the calculation does not address whether this may be due to
tuning or to genuine anisotropy.
For association models, the large Bayes factors for periods 1 and 2,
and the small Bayes factor in Period~3, are not individually surprising.
But it is very surprising to see a combination of large Bayes factors
for the two small subsamples, and a small Bayes factor for the large
subsample.  The full dataset thus is not comfortably fit by either
isotropic or association models.  We discuss this further below.


\section{Summary and Discussion}
\label{sec:summary}

We have described a new multilevel Bayesian framework for modeling the
arrival times, directions, and energies of UHECRs, including statistical
assessment of directional coincidences with candidate sources.
Our framework explicitly models cosmic ray emission, propagation (including
deflection of trajectories by cosmic magnetic fields), and detection.  This
approach cleanly distinguishes astrophysical and experimental processes
underlying the data.  It handles uncertain parameters in these processes via
marginalization, which accounts for uncertainties while allowing use of all
of the data (in contrast to hypothesis testing approaches that optimize over
parameters, requiring holding out a subset of the data for tuning).
We demonstrated the framework by implementing calculations with simple but
astrophysically interesting models for the 69 UHECRs with energies above
55~EeV detected by PAO and reported in PAO-10.  Here we first summarize
our findings based on these models, and then describe directions for
future work.

%Its Bayesian underpinning enables accounting for
%a priori uncertainty in model parameters via averaging (marginalization).
%This stands in contrast to previously-used hypothesis testing approaches
%that handle free parameters defining a test procedure by optimizing using
%a subset of the data, with subsequent analysis omitting the tuning data.

%..............................................................................
\subsection{Astrophysical results}

We modeled UHECRs as coming from either nearby AGN (in a volume-limited
sample including all 17 AGN within 15~Mpc) or an isotropic background
population of sources; AGN are considered to be standard candles in
our models.  We thoroughly explored three models.  In $M_0$ all CRs come
from the isotropic background; in $M_1$ all CRs come from either a
background or one of the 17 closest AGN; in $M_2$ all CRs come from either
a background source or one of the two closest AGN (Cen~A and NGC~5128,
neighboring AGN at a distance of 5~Mpc).  The data were reported in three
periods.  Data from Period~1 were used to tune the energy threshold defining
the published samples in all periods by maximizing an index of anisotropy in
Period~1.  Out of concern that this tuning compromises the data in Period~1
for our analysis, we analyzed the full dataset and various subsamples,
including an ``untuned'' sample omitting Period~1 data.

Using {\em all} of the data, Bayes factors indicate there is strong evidence
favoring either $M_1$ or $M_2$ against $M_0$ but do not discriminate between
$M_1$ and $M_2$.  The most probable models associate about 5\% to 15\% of
UHECRs with nearby AGN, and strongly rule out associating more than
$\approx 25$\% of UHECRs with nearby AGN.  Most of the high-probability
associations in the 17~AGN model are with the two closest AGN.

However, if we use only the {\em untuned} data, the Bayes factors are
equivocal (although the most probable association models resemble those
found using all data).  If we subdivide the untuned data, we find positive
evidence for association using the Period~2 sample, but weak evidence {\em
against} association using the much larger Period~3 sample.  Together, these
results suggest that the statistical character of the data may differ from
period to period, due to tuning of the Period~1 data or other causes.

One way to explore this is to ask whether the data from the various periods
are better explained using models with differing parameter values rather
than a shared set of values.  We investigated this via a change-point
analysis that considered the times bounding the periods as candidate change
points.  The results are consistent with the hypothesis that the parameters
do {\em not} vary between periods, justifying using the combined data for
these models.  This suggests the variation of the Bayes factors across
periods is a consequence of the modest sample sizes.  However, the
change point analysis does not address the possibility that none of the
models is adequate, with model misspecification being the cause of the
apparently discrepant Bayes factors.

\mnote{Significantly altered next three paragraphs}
We used simulated data from both the isotropic model and
high-probability association models to perform predictive checks of our
models, using the Bayes factors based on subsets of the data as test
statistics.  Simulations based on the isotropic model indicate that large
Bayes factors favoring association are unlikely for {\em untuned} samples of
the size of the Period~1 sample.  Simulations based on representative
association models indicate that such Bayes factors are not surprising
for samples of the size of Period~1, considered in isolation.  But
the observed pattern of large Bayes factors for the subsamples in
periods 1 and 2, and a small Bayes factor for the much larger Period~3
subsample, is very surprising.  The full dataset thus is not fit
comfortably by either isotropic models or standard-candle association
models.
Whether the effects of tuning could explain the apparent inconsistencies
remains an open question that is not easy to address without access to the
untuned data.

\mnote{Clarified}
Restricting to the untuned data (periods 2 and 3), the pattern of Bayes
factors is consistent with both isotropic models and representative standard
candle association models.
The best-fitting association models assign a few percent of UHECRs to nearby
AGN; at most $\approx 20$\% may be associated with AGN, with the remainder
assigned to sources drawn from an isotropic distribution.
Magnetic deflection angular scales of $\approx 3^\circ$
to $30^\circ$ are favored.
Models that assign a large fraction of UHECRs to a single nearby source (e.g.,
Cen~A) are ruled out unless very large deflection scales are specified a
priori, and even then they are disfavored.

Even restricting to results based on the untuned data, we hesitate to offer
these models as astrophysically plausible explanations of the PAO UHECR
data, both because of how important the problematic Period~1 sample is in
the analysis, and because of astrophysical limitations of the models
considered here and elsewhere.  In particular, the high-probability models
assign the vast majority of UHECRs to sources in an isotropic distribution.
But the observation by PAO of a GZK-like cutoff in the energy spectrum of
UHECRs argues strongly that UHECRs originate from within $\sim 100$~Mpc,
where the distribution of both visible matter (galaxies) and dark matter is
significantly {\em an}isotropic.  If most or all UHECRs are protons, so that
magnetic deflection is not very strong, an isotropic distribution of UHECR
arrival directions is implausible.  It then may be the case that some of the
strength of the evidence for association with nearby AGN is due to the
``straw man'' nature of the isotropic alternative.  On the other hand, if
most UHECRs are heavy nuclei, then strong magnetic deflection could
isotropize the arrival directions.  The highest probability association
models have relatively small angular deflection scales, but it could be that
the few UHECRs that these models associate with the nearest AGN happen to be
protons or very light nuclei.  Future models could account for this by
allowing a mixture of $\kappa$ values among cosmic rays, as noted in
\S~\ref{sec:dflxn}.

In addition, the standard candle cosmic ray intensity model adopted here and
in other studies is astrophysically implausible and very likely strongly
constrains inferences.  The strongest visible clustering of measured UHECR
directions is toward the two closest AGN, just 5~Mpc away and only a few
degrees apart on the sky.  Most of the high-probability associations
identified in our models are to these AGN.  They are so close that they
imply standard-candle cosmic ray intensities that would produce a negligible
flux of cosmic rays from the vast majority of other AGN within 100~Mpc.  Put
another way, a standard-candle model assigning just one UHECR to an AGN near
the 100~Mpc GZK limit would imply a cosmic ray flux from nearby AGN so huge
that this scenario is ruled out simply by visible inspection of sky maps. 
Models with more flexible luminosity functions would likely allow assignment
of many more UHECRs to sources spanning a range of distances.

%..............................................................................
\subsection{Future directions}

All of these considerations indicate a more thorough exploration of
UHECR production and propagation models is needed.  We thus consider
the analyses here to be a demonstration of the utility and feasibility
of analyzing such models within a multilevel Bayesian framework, and
not a definitive astrophysical analysis of the data.
We are pursuing more complex models elsewhere, expanding on the present
analysis in four directions.

First, we are considering larger, statistically well-characterized catalogs
of potential hosts, e.g., the recently-compiled catalog of X-ray selected
AGN detected by the Burst and Transient (BAT) instrument on the {\em Swift}
satellite, a catalog considered by PAO-10.

Second, we are building more realistic background distributions, for example
by using the locations of nearby galaxy clusters, or the entire nearby
galaxy distribution, to build smooth background densities (e.g., via kernel
density estimation, or fitting of mixture or multipole models).

Third, we are considering richer luminosity function models, including models
assigning a distribution of cosmic ray intensities to all candidate sources,
and models that place some sources in ``on''
states and the others ``off.''  The latter models are motivated both by the
possibility of beaming of cosmic rays, and by evidence for AGN intermittency
in jet substructure, and could enable assignment of significant numbers of
UHECRs to both distant and nearby sources.

Finally, more complicated deflection models are possible.  
For example, we have developed a class of ``radiant'' models that produce
correlated deflections (as seen in some astrophysical simulations).  For a
radiant model, each source has a single guide direction associated with it,
drawn from a Fisher distribution centered at the source direction, with
concentration $\kappa_g$; the guide direction serves as a proxy for the shared
magnetic deflection history of cosmic rays from that source.  Each cosmic ray
associated with that source then has its arrival direction drawn from an
independent Fisher distribution centered about the guide direction, with
concentration potentially depending on cosmic ray energy and source distance;
this distribution describes the effect of the deflection history unique to a
particular cosmic ray.  The resulting directions for a multiplet will cluster
along a ray pointing toward the source.  The resulting joint distribution for
the directions in a multiplet (with the guide direction marginalized) is
exchangeable but not independent.

For the current, modest-sized UHECR catalog, the complexity of
some of these generalizations  is probably not warranted.  But PAO is
expected to operate for many years, and the sample is continually growing in
size.  Making the most of existing and future data will require, not only
more realistic models, but also more complete disclosure of the data.
In particular, a fully Bayesian treatment---including modeling of the energy
dependence in the UHECR flux and deflection scale---requires data
uncorrupted by tuning cuts.  Further, the most accurate analysis should use
event-specific direction and energy uncertainties (likelihood summaries),
rather than the typical error scales currently reported.  We hope 
our framework helps motivate more complete releases of future PAO data.


\appendix

\section{Auger observatory exposure}
\label{app:expo}

PAO can reliably detect and measure UHECRs arriving from directions within
$60^\circ$ from the observatory zenith.  Due to Earth's rotation, the zenith
traces a circular path on the sky, and the PAO field of view changes with time.
The observatory's geodetic latitude is $-35.5^\circ$, so the field of view
always includes the SCP and directions within a $5.5^\circ$ cone about it.
Outside of that cone, the time spent within the field of view decreases with
increasing latitude, vanishing for northern latitudes above $24.5^\circ$.  This
boundary corresponds to the thick gray curve shown in the sky maps.

In addition, at a given instant, the projected area of the observatory varies
with direction.  Since the observatory detects air showers, the effective area
of the detector toward a particular source direction is the projected area of
the layer of atmosphere above PAO toward that direction.  Due to Earth's
rotation, this projected area is a function of time; we denote it by
$A_p(t,\tdrxn)$ for direction $\tdrxn$ (a unit vector toward a fixed direction
on the sky) at time $t$.  We account for the zenith angle criterion by
setting $A_p=0$ for directions outside of the instantaneous field of view.

As described in \S~\ref{sec:data}, the exposure map, $\expo(\tdrxn)$,
is defined by
\begin{equation}
\expo(\tdrxn) \coloneqq \int_T  A_p(\tdrxn,t) dt.
\label{expo-def}
\end{equation}
The projected area can be written as $A_p(\tdrxn,t) = A(t) \mu(\tdrxn,t)$,
where $A(t)$ is the effective planar area of the detection volume, and
$\mu(t,\tdrxn)$ is a projection factor.  $A(t)$ varies slowly with time as
the observatory grows.  The projection factor varies much more quickly, due
to Earth's rotation. The PAO team has shown that for UHECRs, to a good
approximation a simple geometric projection varying periodically with a
period of 1~sidereal day gives a very accurate description of the PAO
exposure \cite{2001APh....14..271S}. As a result, the time integral in
equation~(\ref{expo-def}) can be approximated as $\int A(t) m(\tdrxn) dt$,
where $m(\tdrxn)$ is the geometric projection factor averaged over 1
sidereal day. The average projection factor is constant with respect to
right ascension (the equatorial sky coordinate corresponding to geodetic
longitude) due to rotational averaging.  It varies strongly with declination
(the equatorial sky coordinate corresponding to geodetic latitude).
Figure~\ref{fig:pjxn} shows the average projection as a function of
declination.

\begin{figure}
\centerline{\includegraphics[width=.8\textwidth]{avg_pjxn_factor.eps}}
\caption{Average projection factor, $m(\tdrxn)$, describing the declination
dependence of the PAO exposure map.}
\label{fig:pjxn}
\end{figure}

With these approximations, to evaluate $\expo(\tdrxn)$, we need the
time integral of the observatory area, $A(t)$.  By convention, this
quantity is reported indirectly by describing the observatory's
sensitivity to an isotropic distribution of sources (lower-energy cosmic
rays have an isotropic distribution).  For such a distribution, the expected
number of rays would be proportional to the {\em total exposure}, the
integral of the exposure map over the whole sky:
\begin{equation}
\alpha_T \coloneqq \int \expo(\tdrxn) d\tdrxn = \int A(t) dt \int m(\tdrxn) d\tdrxn.
\label{ET-def}
\end{equation}
The total exposure has units of area $\times$ time $\times$ solid angle;
(it has also been called ``aperture,'' as ``exposure'' is more traditionally
used for quantities with units of area $\times$ time).  The PAO team
reports $\alpha_T$ for each observing period in PAO-10.

To calculate the exposure map from $\alpha_T$ and $m(\tdrxn)$, define the
integrated projection factor by $M \coloneqq \int d\tdrxn m(\tdrxn)$ (with
units of solid angle).  Then the exposure is
\begin{equation}
\expo(\tdrxn) = \frac{\alpha_T}{M} m(\tdrxn).
\label{expo-mET}
\end{equation}


\section{Likelihood factorization}
\label{app:like}

Consider estimating the probability density function for a sample of $N$
points $\{x_i\}$ in a Euclidean space using a mixture model built from the
parameterized kernel density $f(x;\theta)$ with parameters $\theta$
(typically including at least location and scale parameters).  With the
number of components in the mixture fixed as $K$, the likelihood function
for the set of kernel parameters $\{\theta_k\}$ and (normalized) mixing
weights $\{w_k\}$ is
\be
\like(\{w_k\},\{\theta_k\}) = 
  \prod_{i=1}^N \left[ \sum_{k=1}^K w_k f(x_i;\theta_k) \right].
\label{like-mix}
\ee
The factor associated with a particular datum, $\sum_k w_k f(x_i;\theta_k)$,
may be interpreted two ways. We may consider it to be the value of a single,
complex density function that happens to representable as a weighted sum.
Alternatively, we may consider it to represent a choice of one of the $K$
components, with probability $w_k$, as the source for the datum, which is
then drawn from the component density $f(x_i; \theta_k)$.  We can make
the latter interpretation more explicit by introducing latent labels,
$\{\lambda_i\}$, with $\lambda_i = k$ denoting assignment of datum $i$
to component $k$.  To keep track of the probabilities for particular
assignments in equation~(\ref{like-mix}), we use the labels to
distinguish the values of $w_k$ associated with the various datum
factors, by replacing $w_k$ in the $i$th factor with $w_{\lambda_i}$
(and summing over values of $\lambda_i$ rather than $k$).
Abbreviating $f(x_i;\theta_k)$ by $f_{k,i}$, we can rewrite
the likelihood as follows:
\begin{align}
\like(\{w_k\},\{\theta_k\})
  &= \prod_{i=1}^N \left[ \sum_{\lambda_i=1}^{K} w_{\lambda_i}
          {f}_{\lambda_i, i} \right]\\
  &= \left(\sum_{\lambda_1=1}^{K} w_{\lambda_1}
          {f}_{\lambda_1, 1}\right) \times\cdots\times
     \left(\sum_{\lambda_N=1}^{K} w_{\lambda_N}
          {f}_{\lambda_N, N}\right)\nonumber\\
  &= \sum_{\lambda_1=1}^{K} \cdots \sum_{\lambda_N=1}^{K}
     \prod_{i=1}^N w_{\lambda_i} {f}_{\lambda_i, i}\nonumber\\
  &= \sum_{\lambda_1\ldots\lambda_N}
     \left( \prod_k w_k^{m_k(\lambda)}\right)
    \prod_i {f}_{\lambda_i, i},\nonumber
\label{product-sum}
\end{align}
where for the last line we have collected factors of a particular weight by
introducing a multiplicity function, $m_k(\lambda)$, counting the number of
times component index $k$ appears in the list of $N$ labels $\lambda_i$.
This dual representation is well known in the literature on FMMs
(see, e.g., \cite{BG88-BayesFMM}).

Turning now to the cosmic ray likelihood function in
equation~(\ref{like-poisson}), the product factor has the algebraic form of
a FMM likelihood, with the source fluxes, $F_k$, playing the role of
mixing weights (but now no longer normalized).  Equation~(\ref{like-assoc})
then follows by the same manipulations as shown above, with the
addition of splitting the exponentiated sum in equation~(\ref{like-poisson})
into separate $e^{-F_k\epsilon_k}$ factors, grouped with with their
associated $F_k^{m_k(\lambda)}$ factors.  The resulting likelihood
function essentially corresponds to a Poissonized FMM.


\section{Computational methods}
\label{app:compn}


%..............................................................................
\subsection{Algorithm for Markov chain Monte Carlo}
\label{sec:MCMC}

To draw posterior samples, we perform Metropolis-within-Gibbs sampling on
parameters $f,F_T$ and $\lambda$, using Gibbs sampling for $F_T$ and
$\lambda$, and Metropolis sampling for $f$.  The Gibbs sampling steps
alternate between sampling from the full conditional distribution for $F_T$
(i.e., the distribution given the data and all other parameters), and that for
$\lambda$.  The full conditionals may be derived from (\ref{eq:like}) and the
$(F_T,f)$ priors.  The conditional for the total flux is
\be
F_T|f,\lambda,D \sim 
  \text{Gamma}\left(N_C+1,
    \frac{1}{\frac{1}{s}+(1-f)\epsilon_0+f\sum_{k\geq 1}w_k\epsilon_k}\right),
\ee
where $\text{Gamma}(\alpha,s)$ denotes the gamma distribution with shape
parameter $\alpha$ and scale parameter $s$.  (Note that this distribution
happens to be independent of $\lambda$.)  The conditional for the
cosmic ray labels is a multinomial distribution with probabilities
\be
P(\lambda_i|F_T,f,D)
  \propto \frac{f_{\lambda_i,i}}{\epsilon_{\lambda_i}}\times h_{\lambda_i},
    \text{ where } h_{j} =
\begin{cases}(1-f)\epsilon_0 & \text{if $j=0$,}\\
  fw_j\epsilon_j &\text{if $j\geq 1$.}
\end{cases}.
\ee
Finally, the conditional for the associated fraction is
\ba \quad
P(f|\lambda,F_T,D)
  &\propto& \exp\left\{-F_T\left[(1-f)\epsilon_0+f\sum_{k\geq1}\epsilon_k w_k\right]\right\}\nonumber \\
  & & \times (1-f)^{m_0(\lambda)+b-1}f^{N_C-m_0(\lambda)+a-1}.
\ea
$F_T$ and $\lambda$ are sampled directly from the gamma and multinomial
distributions.  $f$ is sampled using a random walk Metropolis algorithm with
Gaussian proposals centered around the current value of $f$.
The variance of the Gaussian proposal density was tuned so that the
acceptance rate was about 25$\%$.

%..............................................................................
\subsection{Marginal Likelihood and Bayes Factor Computation}
\label{sec:Chib}

Following Chib (1995), we can write the marginal likelihood for $\kappa$ as
\be
\like_m(\kappa) = 
  \frac{P\left(D|F_T^*,f^*,\lambda^*\right)
        P\left(\lambda^*|F_T^*,f^*\right)P\left(F_T^*\right)
        P\left(f^*\right)}
       {P\left(F_T^*,f^*,\lambda^*|D\right)} \qquad ||\kappa,
\ee
where the double solidus indicates all the probabilities additionally
condition on $\kappa$.
Here $F_T^*, f^*, \lambda^*$ are in principle arbitrary, but in practice
should correspond to a point with high posterior density.  All the
terms in the numerator can be computed analytically, using the priors and
the likelihood from equation~(\ref{eq:lik-lambda}).  The denominator can be
expressed as
\be
P\left(F_T^*,f^*,\lambda^*|D\right) =
  P(f^*|F_T^*,\lambda^*,D)P(F_T^*|\lambda^*,D)P(\lambda^*|D)
   \qquad ||\kappa.
\ee
The first term on the right hand side is simply the full condition of $f^*$
evaluated at $F_T^*$ and $\lambda^*$.  Note that the normalizing constant can
be computed using numerical integration.  The remaining two terms need to be
estimated using MCMC and can be done as follows:
\be
P(F_T^*|\lambda^*,\kappa,D) \approx
  \frac{1}{G} \sum_{g=1}^G P(F_T^*|f'^{(g)},\lambda^*,\kappa,D),
\ee
\be
P(\lambda^*|\kappa,D) \approx
  \frac{1}{G} \sum_{g=1}^G P(\lambda^*|f^{(g)},F_T^{(g)},\kappa,D).
\ee
Here $G$ denotes the number of iterations.  $f^{(g)}$ and $F_T^{(g)}$ denote
the sample from the MCMC in iteration $g$.  $f'^{(g)}$ is a sample from a new
MCMC run using the full conditionals given earlier with $\lambda$ fixed at
$\lambda^*$ in iteration $g$.

For each $\kappa$ of interest, we first ran 5,000 iterations of Metropolis
within Gibbs to obtain the high posterior density values, $F_T^*, f^*,
\lambda^*$.  Then, for each $\kappa$, 3 additional chains of Gibbs sampling
were run, each with 10,000 iterations.  For subsequent calculations and
plots, the chains were thinned, so that the lag-one autocorrelation is at
most 0.15 for all parameters $F_T, f, \lambda$.

We diagnosed convergence by visually inspecting the 
trace plots from different chains, and by computing the Gelman-Rubin potential 
scale reduction statistic for chains of samples of continuous parameters
such as $f$ and $F_T$.  For the discrete $\lambda$ parameters, we computed
the fraction of time that each cosmic ray is assigned to each source
during every 10 iterations. The diagnostics were done on these fractions
similarly to the continuous parameters.

To validate our algorithms (including our convergence criteria) we developed
an enumerative algorithm that can directly calculate several posterior
quantities for simplified models of small catalogs in the small-deflection
regime, via a guided traversal of a tree of possible associations that has
been thresholded to eliminate associations with negligible probability.  We
compared results from this deterministic algorithm with our MCMC results. We
also used simulated datasets to verify that marginal distributions produce
credible intervals of probability $P$ that have prior-averaged coverage
equal to $P$, a simplified version of the validation tests proposed by Cook,
Gelman, and Rubin \cite{CGR06-Validn}.



\section{Cen~A single-source model}
\label{app:CenA}

Recent PAO results on the chemical composition of UHECRs, cited above,
suggest that UHECRs may be predominantly heavy nuclei, which would
suffer large magnetic deflections.  Some investigators have suggested
that UHECRs are all heavy nuclei from a single source---the nearest AGN,
Cen~A---with the apparent approximate isotropy of arrival directions
a consequence of strong deflection \cite{B+09-CenA,GBdS10-CenA,BdS12-CenA}.

As a simple test of this idea, we studied a model corresponding to our
buckshot AGN association model, but with a single candidate source, Cen~A, and
with the association fraction $f=1$.  Figure~\ref{fig:BF-CenA} shows the Bayes
factor comparing such a model to the isotropic background model, conditional
on $\kappa$, for various partitions of the data.  For $\kappa=0$, the Cen~A
model makes the same predictions as the background model, and the Bayes factor
is unity.  As $\kappa$ increases, the Bayes factor never grows significantly
larger than unity, so the Cen~A model is not supported by any partition of the
data, for any $\kappa$.  For partitions of the data including the 42 Period~3
events, the Bayes factor opposes the Cen~A model, strongly ruling out models
with $\kappa\simgreat 0.5$, i.e., with deflection angular scales $<90^\circ$.

\begin{figure}
\centerline{\includegraphics[angle=-90,width=\textwidth]{BF-CenAvsIsotropic-SmallKappa.eps}}
\caption{Bayes factors comparing a model Cen~A to be the source of
all UHECRs against the null isotropic background model, conditional
on $\kappa$, shown as a function of $\kappa$.  Results are
shown for various partions of the data (identified by line style,
identified in the legend).  Right panel expands the low-$\kappa$ region.}
\label{fig:BF-CenA}
\end{figure}

The $90^\circ$ scale is consistent with the expected angular scale for
deflection by a regular magnetic field in equation~(\ref{dflxn-reg}) as long
as $Z$ is large ($\simgreat 15$); but note that this equation is valid only in
the small-deflection limit.  The $90^\circ$ scale is consistent with turbulent
magnetic deflection only for the heaviest nuclei (e.g., for iron $Z=26$) and
for large magnetic field and length scales.  For these astrophysical
parameters, the Cen~A model remains disfavored, although not overwhelmingly. 
We thus consider the PAO-10 data to argue against the hypothesis that all
UHECRs originate from Cen~A.


%..............................................................................
\section{Comparison with prior Bayesian work}
\label{app:WMJ11}

Watson, Mortlock, and Jaffe \cite{WMJ11-BayesUHECR} (WMJ11) have
independently developed an approximate Bayesian approach for assessing
evidence for association of the PAO-08 data (with data for the 27 UHECRs in
periods 1 and 2) with nearby AGN.  They consider the best-fit CR directions
as points drawn from a Poisson intensity on the celestial sphere constructed
as a mixture of distributions located at AGN directions, with weights
reflecting the AGN distances (corresponding to a standard-candle
source intensity), and an isotropic background component.  The source
components have the form of a ``two-dimensional Gaussian on the sphere,''
with the probability density for a best-fit measured direction $\hat n$
arising from an AGN at $\hdrxn$ given by
\be
p(\hat n|\hdrxn)
  = \frac{1}{2\pi\left(1 - e^{-2/\sigma^2}\right)}
    \exp\left(-\frac{1 - \hat n\cdot\hdrxn}{\sigma^2}\right),
\label{sphere-gauss}
\ee
with $\sigma$ dubbed the ``smearing angle.'' This is a Fisher distribution
(as in equation~\ref{rho-def}), with concentration parameter $\kappa =
1/\sigma^2$.  They focus on a model with  $\sigma = 3^\circ$ ($\approx
0.052$~rad) [sic], intended to reflect a combination of $\approx 1^\circ$
measurement uncertainties and few-degree magnetic deflections.  They also
provide a few results for $\sigma = 6^\circ$ and $10^\circ$ [sic].\footnote{We
note that the parameter $\sigma$ is not an angle and does not have angular
units (degrees or radians).  We presume these dimensionally inconsistent
equations imply solving the nonlinear equation relating $\sigma$ (or $\kappa$)
to the stated angular scale; e.g., for a 68.3\% confidence or credible region
with angular radius $\theta$, in the small-angle limit $\sigma \approx
0.66\theta$ (e.g., $\sigma \approx 0.035$ for $3^\circ$ uncertainties;
see equation~(\ref{kappa-theta})).}
They calculate an approximate likelihood function by finely pixelizing the
celestial sphere, calculating the number of cosmic ray direction
measurements expected in each pixel, and multiplying Poisson counting
probabilities for the bins (with one count for bins containing a best-fit
cosmic ray direction, and zero counts for the remaining bins).

WMJ11 adopt a similar candidate host population as was used in the PAO-07 and
PAO-08 analyses (the nearby AGN in the 12th VCV compilation; WMJ11 consider
$\approx 900$ AGN within 100~Mpc) on the presumption that it is almost
complete for the nearest AGN.  They conclude that there is ``strong evidence
of a UHECR signal from the known VCV AGNs,'' such that at least some UHECRs
come from AGN in the VCV catalog (or from sources within a few degrees of the
AGN).  For a $3^\circ$ smearing angle, the marginal posterior density for the
fraction associated with AGN (our $f$ parameter) has a mode of 0.15, and 68\%
highest density credible interval of $[0.08, 0.25]$.  For $6^\circ$ and
$10^\circ$ smearing angles the estimated association fraction is larger, but
the marginal distribution is also somewhat broader.  They do not compare their
association model to an alternative, and thus do not compute Bayes factors.

Our analysis framework and our results differ in significant ways from those
of WMJ11 (focusing on our results for the data from Periods~1 and 2). 
Methodologically, our approach is based on explicit modeling of associations
(via marginal likelihood factors associating a particular cosmic ray with a
particular AGN or the background), rather than considering best-fit cosmic
ray directions to be samples from a point process.  As noted in
\S~\ref{sec:dtxn}, a factor in the likelihood function in our approach
is analogous to that underlying FMMs, but this
mixture-like factor arises as a consequence of some of our modeling
assumptions; it is not a starting point, and it does not hold for all
astrophysically interesting association models that our framework
accommodates.  For example, in \S~\ref{sec:summary} we describe a more
realistic family of magnetic deflection models (``radiant'' models with
exchangeable rather than IID deflections for cosmic rays comprising a
multiplet) whose likelihood function does not have the simple FMM form.  In
other common astronomical coincidence assessment problems, such as
establishing associations of sources detected in different electromagnetic
wavebands, only singlet associations are meaningful; such models have no FMM
representation but may be accommodated by our framework.  Further
discussion of this is in \cite{Loredo12-Coinc}.

Another point of departure in methodology is that we distinguish measurement
error from magnetic deflection.  In the buckshot model adopted here, both
the measurement error and deflection distributions are Fisher distributions
(note that the composition of these distributions is not a Fisher
distribution).  In radiant models, for example, the deflection distribution
is more complicated.  Explicit, separate treatment of these physically
distinct effects enables handling heteroskedastic measurement errors
for the UHECR directions (PAO-10 reports only a typical measurement
error, but future catalogs will hopefully report the heteroskedastic
uncertainties found in detailed air shower fits).  Heteroskedastic
uncertainties further thwart a simple FMM representation for the likelihood,
again emphasizing the need for a framework built on explicit modeling
of individual associations.

Our approach also provides explicit estimates of probabilities for possible
associations.  In our MCMC algorithm, these can be found by calculating
frequencies for different values of the $\lambda$ labels; we report
such results in \S~\ref{sec:results}.  WMJ11 report a weight for
candidate associations, but note it is not a rigorous probability.

WMJ11 analyze the data from Periods~1 and 2 jointly, without commenting on the
possible effects of tuning on the implications of the Period~1 data.

Turning to astrophysical differences, we adopt the G10 volume-complete catalog
of nearby AGN as a candidate host catalog, rather than the 12th VCV catalog
used by WMJ11 and others.  Notably, 6 of the 17 AGN in the G10 catalog are not
in the 12th VCV catalog (one of them appears in the more recent 13th VCV
catalog).  We chose the G10 catalog both for its completeness, and because use
of a small catalog was convenient for an initial study, as it enabled more
extensive analyses of real and simulated data than would be possible with AGN
from VCV. Figure~\ref{fig:f1000} shows our marginal posterior density for $f$
for $\kappa = 1000$ (corresponding to a deflection scale $\approx 2.7^\circ$)
for model $M_1$, for different subsets of the PAO-10 data. The mode based on
the combined data from Periods~1 and 2 is at $f \approx 0.1$, about a
two-thirds of the WMJ11 value of $\approx 0.15$, even though their catalog
contains more than 50 times as many potential counterparts.  Our common
assumption of a standard candle intensity distribution is probably the main
reason that the results are not more discrepant. In particular, although WMJ11
include AGN at distances to 100~Mpc in their catalog, the standard candle
assumption forces the analysis to assign negligible detectable rates to all
but the closest few AGN.  The similarity of our estimates despite the
disparity between our AGN catalog sizes highlights how restrictive the
standard candle assumption is.

\begin{figure}
\centerline{\includegraphics[angle=-90,width=.9\textwidth]{margf_kappa1000_17AGNs.eps}}
\caption{Posterior distributions for $f$ for model $M_1$, conditioned on
$\kappa$ = 1000, for various subsets of the PAO-10 data.}
\label{fig:f1000}
\end{figure}

We explore a far greater range of magnetic deflection angular scales
than did WMJ11.  Their discussion of deflection scales implicitly
presumes UHECRs are light nuclei.  Recent cosmic ray data and
theoretical models motivate serious consideration of the possibility
that many or most UHECRs are heavy nuclei, as noted above.

Finally, we differ qualitatively in our conclusions about the strength of
evidence for association of UHECRs with nearby AGN, particularly after
examining period-to-period differences, and considering the impact of
period~3 data (unavailable to WMJ11).  We quantify the support for
association hypotheses by calculating Bayes factors explicitly comparing
association and null models, both conditional on $\kappa$ (in
Figure~\ref{fig:BFplot}) and marginalized over a broad $\kappa$
range (in Table~\ref{BFTable}).

WMJ11 do not calculate Bayes factors comparing their association and null
models.  Their claim of strong evidence for association appears to be based on
the small marginal posterior density for values of the association fraction
near zero.  But this fails to distinguish parameter estimation from
model assessment.  As long as the likelihood for models with $f=0$ does not
vanish, parameter estimation (with a continuous prior on $f$) simply does not
address how the $f=0$ hypothesis compares to alternatives.  To do this
requires assigning a finite prior probability for $f=0$, which we do here by
considering it as a separate model and calculating Bayes factors.  The Bayes
factor comparing nested models depends on the size of the parameter space of
the larger model in a way that accounts for ``fine tuning'' of the additional
model parameters:  the larger model will have parameter values producing
better fits than the smaller model, but if the values of the additional
parameters are close enough to the default values corresponding to the smaller
model, the marginal likelihood for the larger model will be {\em smaller} than
that for the smaller model (the well-known ``Ockham's razor'' behavior of
Bayes factors).  Focusing on low-dimensional marginal distributions, such as
the posterior density for $f$, can give an exaggerated impression of the
strength of evidence for the larger model because it suppresses the large
volume of parameter space associated with its additional parameters.  Here,
association models have not only the $f$ parameter, but also many latent label
parameters (i.e., many association hypotheses that cannot be ruled out a
priori).  Calculation of the Bayes factor takes all of this into account.
Using the Period~1 and Period~2 data available to WMJ11 does in fact produce
large Bayes factors favoring association.  But partitions of the data
excluding Period~1 data produce much smaller Bayes factors, even though the
$p(f)$ distributions found with these partitions assign very small density to
$f=0$. The Bayes factor calculations indicate that the complexity of
association models may not be justified by existing data.

Perhaps most importantly from an astrophysical perspective, we performed
more extensive checking of our models, calling into question the adequacy of
our shared isotropic background and standard candle assumptions.  We discuss
this further in \S~\ref{sec:summary}.



\section{Model checking}
\label{app:checking}

In this Appendix we describe results of two types of tests of our models.
Both are motivated by the evident variability of some of the parameter
estimation and model comparison results presented in \S~\ref{sec:results}
with the choice of observing period or periods to include in the analysis.

Our first tests address whether the variability indicates that the
properties of the detected cosmic rays change from period to period,
presuming that one of our models can adequately describe the data within
each period.  We implement a simple Bayesian change-point analysis that shows
there is no significant evidence for variability of model parameters from
period to period.  That is, presuming one of the models is adequate, the
apparent discrepancy among the Bayes factors in Table~\ref{BFTable} reflects
variability that may be expected for these modest sample sizes.

Alternatively, we may consider the possibility that none of the considered
models accurately describes the data, in which case the variability could be
an indication of incompatibility of the data with {\em all} of the models.
To address this, we perform predictive checks using the Bayes factors for
subsamples of the data as test statistics:  we ask whether the
period-to-period Bayes factor variations we find using the observed data
are surprising compared with what one finds from simulations under various
models.  We first compare the observed Bayes factors with predictions from the
null model, and then from representative association models.  These tests
are meant to explore broad compatibility of predictions and
observations; we make no attempt to formally assign significances to the
comparison such as $p$-values.

%..............................................................................
\subsection{Change-point analysis}

As seen above, including the data from Period~1 can change Bayes factors
dramatically (but not estimates of $F_T$, $f$, or $\kappa$); also, Bayes
factors differ markedly even between the {\em untuned} samples, Periods~2
and 3.  Based on astrophysical considerations, the properties of incident
UHECRs should not vary over the time scales under consideration.  Even if CRs
are generated in brief bursts, these bursts would have observed durations of
hundreds or thousands of years because the cosmic rays would take paths of
varying lengths to Earth.  Therefore, any evidence indicating that the
observed properties of cosmic rays are changing over a period of a few years
would indicate a problem with the data, e.g., statistical inhomogeneity due to
the special treatment of data in period~1, or instability of the observatory's
apparatus or data reduction pipeline.

As a simple test for variability in the ensemble properties of cosmic rays
from period to period, we compared versions of $M_1$ and $M_2$ that allow
model parameters to change between periods to versions that keep the
parameters the same for all periods.  That is, we
explore change-point models, with the change point locations at period
boundaries.  Specifically, we compute 3 quantities:
\ba
B_{(1)(23)} &=& \frac{\like_1 \like_{23}}{\like_{123}}\nonumber\\
B_{(2)(3)} &=& \frac{\like_2 \like_3}{\like_{23}}\nonumber\\
B_{(1)(2)(3)} &=& \frac{\like_1\like_2 \like_3}{\like_{123}}
\ea
where $\like_{i_1,\ldots,i_q}$ is the marginal likelihood computed using
the Chib estimate based on data from periods $i_1,\ldots,i_q$.
For example, $B_{(1)(23)}$ compares a model allowing parameters to differ
between period~1 and the later periods, to a model with common parameter
values across all periods.  Note that $B_{(2)(3)}$ considers only untuned
data.  We compute $B_{(1)(23)}, B_{(2)(3)}$ and $B_{(1)(2)(3)}$ for both the
17 AGN and 2 AGN association models. Figure~\ref{fig:changepoint} shows
these Bayes factors as functions of $\kappa$.

\begin{figure}
\centerline{$
\begin{array}{cc}
\includegraphics[angle=-90,width=.5\textwidth]{BF_changepoint_17AGNs.eps} &
\includegraphics[angle=-90,width=.5\textwidth]{BF_changepoint_2AGNs.eps}
\end{array}$}
\caption{Bayes factors for change-point models.}
\label{fig:changepoint}
\end{figure}

In the case of 17 AGN, for change-point models considering all of the data,
the Bayes factors stay within [1/3,3] indicating no preference for one model
over the other.
The same is true for the 2~AGN model, except for
$\kappa > 300$, where there is a modest preference for models with
consistent parameter values across all periods.

We also considered change point models that partition the data between
(joint) association models and the null model, aiming to assess the
possibility that the data are consistent with isotropy in some intervals,
but with association (possibly spurious) in others.  We again found
Bayes factors to be equivocal.  

%..............................................................................
\subsection{Null model predictive checks}

If the null model is in operation, a measure of surprise for a particular data
set would be a high Bayes factor favoring an association model.  Of particular
concern here are the large Bayes factor values we find for the Period~1 data,
which may not be representative because of tuning.  We ask:  under the null
isotropic background model and in the absence of tuning, how likely it is to
see Bayes factors as high as 100 for some values of $\kappa$ in Period~1 (as
shown in Figure~\ref{fig:BFplot})?

To address this, we generate 200 datasets, each of which has 14 CRs (the
size of the Period~1 sample).  The CR
directions are generated uniformly over the sky, and are accepted with
probability proportional to the exposure map for that direction.  We
calculated Bayes factors for the 17~AGN association model with $\kappa = 10,
31.62, 100, 316.2$ and 1000.
Figure~\ref{fig:unifCumBF} shows cumulative histograms (as survival
functions) of the resulting Bayes factors, for each value of $\kappa$.  Most
of the datasets have Bayes factors less than 1. The smallest Bayes factor we
found is 0.04. Out of these 200 datasets, we found only 3 datasets with
$B_{10} \geq 10$.  Each of these datasets has $B_{10} \geq 10$ only at one
value of $\kappa$ that we computed. The Bayes factors are 29 ($\kappa=10$),
44 ($\kappa = 31.62$) and 13 ($\kappa = 316.2$).  We conclude that the Bayes
factor greater than 100 seen in Period~1 is unlikely to be due to chance if
UHECRs are fairly sampled from isotropically distributed directions.  This
implies the distribution of directions in the Period~1 sample is
anisotropic, but the calculation does not address whether this may be due to
tuning or to genuine anisotropy.

\begin{figure}
\centerline{\includegraphics[angle=-90,width=\textwidth]{BF_cumplot_genUnif_14CRs_logscale.eps}}
\caption{The number of simulated datasets having Bayes factor $\geq B$
vs. $B$, based on 200 simulated datasets with 14 CRs generated under the
null isotropic background model.  Axes are logarithmic.}
\label{fig:unifCumBF}
\end{figure}

%..............................................................................
\subsection{Association model predictive checks}

Now we address compatibility of the discrepant Bayes factors with the
predictions of the association models.  We ask: under the association model
with 17 AGNs, how likely is it to simultaneously see large Bayes factors in
periods 1 or 2 (which have small numbers of detected CRs) and a small Bayes
factor in Period~3 (which has a larger number of detected CRs)?

To address this, we generate 100 datasets, each with 69 CRs, partitioned
into samples of size 14, 13 and 42 for periods 1, 2 and 3, respectively.  We
simulate from a model with association fraction $f=0.1$, near the modes
found in our analyses of the PAO data.  We first simulate an incident flux
of cosmic rays by assigning a ray to the AGN population with probability
equal to $f$, or to the isotropic background with probability $1-f$.
AGN-generated CRs get assigned to one of the 17 AGN with probabilities
proportional to the inverse square distances of the AGN. The arrival
directions of these CRs are drawn from a Fisher distribution centered at the
source AGN, with $\kappa = 50$.  Each generated CR direction is accepted
with probability proportional to the exposure map for that direction; a
dataset is generated when 69 candidate events are accepted.  Subsets
of each simulated dataset, with sizes corresponding to those of the
PAO subsamples, were analyzed with models conditioning on various
values of $\kappa$.  Figure~\ref{fig:assocCumBF} shows cumulative
histograms of the resulting Bayes factors.

Out of 100 simulated datasets, we found 17 that have some of the Bayes
factors $\geq 30$.  Of these, 9 datasets have $B_{10}\geq 30$ in either
Period~1 or 2, and a low $B_{10}$ value in Period~3, while the other 8
datasets have $B_{10}\geq 30$ in Period~3 and low $B_{10}$ values in periods
1 and 2.  Out of these 17 datasets, 10 of them have some values of
$B_{10}\geq 100$; two have Bayes factors over 1000 (in either Period~1
or Period~2).  
\mnote{Clarified results}
These results indicate that Bayes factors as high as 100 (as seen in Period~1)
may be expected a few percent of the time under the association model, and
also that large between-period variations in the Bayes factors should be
expected.  However, since Bayes factors $\gta 10$ are uncommon, and since the
distribution is independent from one interval to the next, it is unlikely to
see significantly large Bayes factors in {\em both} periods 1 and 2.  We
crudely estimate the probability for seeing Bayes factors whose product
is $> 1000$ as less than $\sim 10^{-3}$.  Thus the pattern of Bayes factors is
surprising, again indicating that the tuning of the Period~1 data is
problematic.

\begin{figure}
\centerline{\includegraphics[angle=-90,width=\textwidth]{BFCumplot-AssocnSimn.eps}}
\caption{The number of simulated datasets having Bayes factor $\geq B$
vs. $B$, based on 100 simulated datasets with CRs generated under the
17-AGN association model with $f=0.1$ and $\kappa=50$.  Panels show
results for datasets with sizes corresponding to the PAO samples for
periods 1--3, as labeled; colors of curves indicate the $\kappa$ value
used for analyzing the simulated data.  Axes are logarithmic.}
\label{fig:assocCumBF}
\end{figure}



\section*{Acknowledgements}
This research was supported by the NSF via grants AST-0908439 and DMS-0805975.
We are grateful to Paul Sommers for helpful conversations about the
PAO instrumentation and data reduction and analysis process.

% Descriptions of online supplements go here:

%\begin{supplement}[id=suppA]
%  \sname{Supplement A}
%  \stitle{Title}
%  \slink[url]{http://lib.stat.cmu.edu/aoas/???/???}
%  \sdescription{Some text}
%\end{supplement}


\bibliographystyle{plain}
\newcommand{\apj}{Astrophysical Journal}
\newcommand{\apjl}{Astrophysical Journal Letters}
\newcommand{\mnras}{Monthly Notices of the Royal Astronomical Society}
\newcommand{\jcap}{Journal of Cosmology and Astroparticle Physics}

\bibliography{UHECR}

% AOS,AOAS: If there are supplements please fill:
%\begin{supplement}[id=suppA]
%  \sname{Supplement A}
%  \stitle{Title}
%  \slink[url]{http://lib.stat.cmu.edu/aoas/???/???}
%  \sdescription{Some text}
%\end{supplement}

\end{document}

Chib, S. (1995) Marginal likelihood from the Gibbs Output, JASA, 90,
1313—1321.

Newton, M. A., and Raftery, A. E. (1994) Approximate Bayesian inference by the
weighted likelihood bootstrap, JRSS-B, 56, 3-48.

Cook, Gelman, Rubin 2006

